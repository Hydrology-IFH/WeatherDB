%% Generated by Sphinx.
\def\sphinxdocclass{report}
\documentclass[letterpaper,10pt,english]{sphinxmanual}
\ifdefined\pdfpxdimen
   \let\sphinxpxdimen\pdfpxdimen\else\newdimen\sphinxpxdimen
\fi \sphinxpxdimen=.75bp\relax
\ifdefined\pdfimageresolution
    \pdfimageresolution= \numexpr \dimexpr1in\relax/\sphinxpxdimen\relax
\fi
%% let collapsible pdf bookmarks panel have high depth per default
\PassOptionsToPackage{bookmarksdepth=5}{hyperref}

\PassOptionsToPackage{warn}{textcomp}
\usepackage[utf8]{inputenc}
\ifdefined\DeclareUnicodeCharacter
% support both utf8 and utf8x syntaxes
  \ifdefined\DeclareUnicodeCharacterAsOptional
    \def\sphinxDUC#1{\DeclareUnicodeCharacter{"#1}}
  \else
    \let\sphinxDUC\DeclareUnicodeCharacter
  \fi
  \sphinxDUC{00A0}{\nobreakspace}
  \sphinxDUC{2500}{\sphinxunichar{2500}}
  \sphinxDUC{2502}{\sphinxunichar{2502}}
  \sphinxDUC{2514}{\sphinxunichar{2514}}
  \sphinxDUC{251C}{\sphinxunichar{251C}}
  \sphinxDUC{2572}{\textbackslash}
\fi
\usepackage{cmap}
\usepackage[T1]{fontenc}
\usepackage{amsmath,amssymb,amstext}
\usepackage{babel}



\usepackage{tgtermes}
\usepackage{tgheros}
\renewcommand{\ttdefault}{txtt}



\usepackage[Bjarne]{fncychap}
\usepackage{sphinx}

\fvset{fontsize=auto}
\usepackage{geometry}


% Include hyperref last.
\usepackage{hyperref}
% Fix anchor placement for figures with captions.
\usepackage{hypcap}% it must be loaded after hyperref.
% Set up styles of URL: it should be placed after hyperref.
\urlstyle{same}

\addto\captionsenglish{\renewcommand{\contentsname}{Contents:}}

\usepackage{sphinxmessages}
\setcounter{tocdepth}{2}


% Jupyter Notebook code cell colors
\definecolor{nbsphinxin}{HTML}{307FC1}
\definecolor{nbsphinxout}{HTML}{BF5B3D}
\definecolor{nbsphinx-code-bg}{HTML}{F5F5F5}
\definecolor{nbsphinx-code-border}{HTML}{E0E0E0}
\definecolor{nbsphinx-stderr}{HTML}{FFDDDD}
% ANSI colors for output streams and traceback highlighting
\definecolor{ansi-black}{HTML}{3E424D}
\definecolor{ansi-black-intense}{HTML}{282C36}
\definecolor{ansi-red}{HTML}{E75C58}
\definecolor{ansi-red-intense}{HTML}{B22B31}
\definecolor{ansi-green}{HTML}{00A250}
\definecolor{ansi-green-intense}{HTML}{007427}
\definecolor{ansi-yellow}{HTML}{DDB62B}
\definecolor{ansi-yellow-intense}{HTML}{B27D12}
\definecolor{ansi-blue}{HTML}{208FFB}
\definecolor{ansi-blue-intense}{HTML}{0065CA}
\definecolor{ansi-magenta}{HTML}{D160C4}
\definecolor{ansi-magenta-intense}{HTML}{A03196}
\definecolor{ansi-cyan}{HTML}{60C6C8}
\definecolor{ansi-cyan-intense}{HTML}{258F8F}
\definecolor{ansi-white}{HTML}{C5C1B4}
\definecolor{ansi-white-intense}{HTML}{A1A6B2}
\definecolor{ansi-default-inverse-fg}{HTML}{FFFFFF}
\definecolor{ansi-default-inverse-bg}{HTML}{000000}

% Define an environment for non-plain-text code cell outputs (e.g. images)
\makeatletter
\newenvironment{nbsphinxfancyoutput}{%
    % Avoid fatal error with framed.sty if graphics too long to fit on one page
    \let\sphinxincludegraphics\nbsphinxincludegraphics
    \nbsphinx@image@maxheight\textheight
    \advance\nbsphinx@image@maxheight -2\fboxsep   % default \fboxsep 3pt
    \advance\nbsphinx@image@maxheight -2\fboxrule  % default \fboxrule 0.4pt
    \advance\nbsphinx@image@maxheight -\baselineskip
\def\nbsphinxfcolorbox{\spx@fcolorbox{nbsphinx-code-border}{white}}%
\def\FrameCommand{\nbsphinxfcolorbox\nbsphinxfancyaddprompt\@empty}%
\def\FirstFrameCommand{\nbsphinxfcolorbox\nbsphinxfancyaddprompt\sphinxVerbatim@Continues}%
\def\MidFrameCommand{\nbsphinxfcolorbox\sphinxVerbatim@Continued\sphinxVerbatim@Continues}%
\def\LastFrameCommand{\nbsphinxfcolorbox\sphinxVerbatim@Continued\@empty}%
\MakeFramed{\advance\hsize-\width\@totalleftmargin\z@\linewidth\hsize\@setminipage}%
\lineskip=1ex\lineskiplimit=1ex\raggedright%
}{\par\unskip\@minipagefalse\endMakeFramed}
\makeatother
\newbox\nbsphinxpromptbox
\def\nbsphinxfancyaddprompt{\ifvoid\nbsphinxpromptbox\else
    \kern\fboxrule\kern\fboxsep
    \copy\nbsphinxpromptbox
    \kern-\ht\nbsphinxpromptbox\kern-\dp\nbsphinxpromptbox
    \kern-\fboxsep\kern-\fboxrule\nointerlineskip
    \fi}
\newlength\nbsphinxcodecellspacing
\setlength{\nbsphinxcodecellspacing}{0pt}

% Define support macros for attaching opening and closing lines to notebooks
\newsavebox\nbsphinxbox
\makeatletter
\newcommand{\nbsphinxstartnotebook}[1]{%
    \par
    % measure needed space
    \setbox\nbsphinxbox\vtop{{#1\par}}
    % reserve some space at bottom of page, else start new page
    \needspace{\dimexpr2.5\baselineskip+\ht\nbsphinxbox+\dp\nbsphinxbox}
    % mimick vertical spacing from \section command
      \addpenalty\@secpenalty
      \@tempskipa 3.5ex \@plus 1ex \@minus .2ex\relax
      \addvspace\@tempskipa
      {\Large\@tempskipa\baselineskip
             \advance\@tempskipa-\prevdepth
             \advance\@tempskipa-\ht\nbsphinxbox
             \ifdim\@tempskipa>\z@
               \vskip \@tempskipa
             \fi}
    \unvbox\nbsphinxbox
    % if notebook starts with a \section, prevent it from adding extra space
    \@nobreaktrue\everypar{\@nobreakfalse\everypar{}}%
    % compensate the parskip which will get inserted by next paragraph
    \nobreak\vskip-\parskip
    % do not break here
    \nobreak
}% end of \nbsphinxstartnotebook

\newcommand{\nbsphinxstopnotebook}[1]{%
    \par
    % measure needed space
    \setbox\nbsphinxbox\vbox{{#1\par}}
    \nobreak % it updates page totals
    \dimen@\pagegoal
    \advance\dimen@-\pagetotal \advance\dimen@-\pagedepth
    \advance\dimen@-\ht\nbsphinxbox \advance\dimen@-\dp\nbsphinxbox
    \ifdim\dimen@<\z@
      % little space left
      \unvbox\nbsphinxbox
      \kern-.8\baselineskip
      \nobreak\vskip\z@\@plus1fil
      \penalty100
      \vskip\z@\@plus-1fil
      \kern.8\baselineskip
    \else
      \unvbox\nbsphinxbox
    \fi
}% end of \nbsphinxstopnotebook

% Ensure height of an included graphics fits in nbsphinxfancyoutput frame
\newdimen\nbsphinx@image@maxheight % set in nbsphinxfancyoutput environment
\newcommand*{\nbsphinxincludegraphics}[2][]{%
    \gdef\spx@includegraphics@options{#1}%
    \setbox\spx@image@box\hbox{\includegraphics[#1,draft]{#2}}%
    \in@false
    \ifdim \wd\spx@image@box>\linewidth
      \g@addto@macro\spx@includegraphics@options{,width=\linewidth}%
      \in@true
    \fi
    % no rotation, no need to worry about depth
    \ifdim \ht\spx@image@box>\nbsphinx@image@maxheight
      \g@addto@macro\spx@includegraphics@options{,height=\nbsphinx@image@maxheight}%
      \in@true
    \fi
    \ifin@
      \g@addto@macro\spx@includegraphics@options{,keepaspectratio}%
    \fi
    \setbox\spx@image@box\box\voidb@x % clear memory
    \expandafter\includegraphics\expandafter[\spx@includegraphics@options]{#2}%
}% end of "\MakeFrame"-safe variant of \sphinxincludegraphics
\makeatother

\makeatletter
\renewcommand*\sphinx@verbatim@nolig@list{\do\'\do\`}
\begingroup
\catcode`'=\active
\let\nbsphinx@noligs\@noligs
\g@addto@macro\nbsphinx@noligs{\let'\PYGZsq}
\endgroup
\makeatother
\renewcommand*\sphinxbreaksbeforeactivelist{\do\<\do\"\do\'}
\renewcommand*\sphinxbreaksafteractivelist{\do\.\do\,\do\:\do\;\do\?\do\!\do\/\do\>\do\-}
\makeatletter
\fvset{codes*=\sphinxbreaksattexescapedchars\do\^\^\let\@noligs\nbsphinx@noligs}
\makeatother



\title{WeatherDB}
\date{Feb 15, 2022}
\release{0.0.2}
\author{Max Schmit}
\newcommand{\sphinxlogo}{\vbox{}}
\renewcommand{\releasename}{Release}
\makeindex
\begin{document}

\pagestyle{empty}
\sphinxmaketitle
\pagestyle{plain}
\sphinxtableofcontents
\pagestyle{normal}
\phantomsection\label{\detokenize{index::doc}}



\chapter{WeatherDB \sphinxhyphen{} module}
\label{\detokenize{README:weatherdb-module}}\label{\detokenize{README::doc}}
\sphinxAtStartPar
author: Max Schmit

\sphinxAtStartPar
\sphinxhref{https://weatherdb-module.readthedocs.io/en/latest/?badge=latest}{\sphinxincludegraphics[width=86\sphinxpxdimen,height=20\sphinxpxdimen]{{D:\Projekte\WeatherDB\python\module\docs\latex\.doctrees\images\829a3ad0568be832441c08050b58c2b844fef3d7\faf4f515804325b4dd93047c0b69d1ba3c92e8fe}.svg}}

\sphinxAtStartPar
The weather\sphinxhyphen{}DB module offers an API to interact with the automaticaly filled weather Database.

\sphinxAtStartPar
Depending on the Database user privileges you can use more or less methodes of the classes.

\sphinxAtStartPar
There are 3 different sub modules with their corresponding classes.
\begin{itemize}
\item {} 
\sphinxAtStartPar
station:
Has a class for every type of station. E.g. PrecipitationStation (or StationN).
One object represents one Station with one parameter.
This object can get used to get the corresponding timeserie.
There is also a StationGroup class that groups the three parameters precipitation, temperature and evapotranspiration together for one station. If one parameter is not available this one won’t get groupped.

\item {} 
\sphinxAtStartPar
stations:
Is a grouping class for all the stations of one measurement parameter. E.G. PrecipitationStations (or StationsN).
Can get used to do actions on all the stations.

\item {} 
\sphinxAtStartPar
broker:
This submodule has only one class Broker. This one is used to do actions on all the stations together. Mainly only used for updating the DB.

\end{itemize}


\section{Get started}
\label{\detokenize{README:get-started}}
\sphinxAtStartPar
To get started you need to enter the credentials to access the Database. If this is an account with read only acces, than only those methodes, that read data from the Database are available.
Enter those credentials in the secretSettings.py file.


\chapter{weatherDB}
\label{\detokenize{modules:weatherdb}}\label{\detokenize{modules::doc}}

\section{weatherDB package}
\label{\detokenize{weatherDB:weatherdb-package}}\label{\detokenize{weatherDB::doc}}

\subsection{weatherDB.broker module}
\label{\detokenize{weatherDB:module-weatherDB.broker}}\label{\detokenize{weatherDB:weatherdb-broker-module}}\index{module@\spxentry{module}!weatherDB.broker@\spxentry{weatherDB.broker}}\index{weatherDB.broker@\spxentry{weatherDB.broker}!module@\spxentry{module}}\index{Broker (class in weatherDB.broker)@\spxentry{Broker}\spxextra{class in weatherDB.broker}}

\begin{fulllineitems}
\phantomsection\label{\detokenize{weatherDB:weatherDB.broker.Broker}}\pysigline{\sphinxbfcode{\sphinxupquote{class\DUrole{w}{  }}}\sphinxcode{\sphinxupquote{weatherDB.broker.}}\sphinxbfcode{\sphinxupquote{Broker}}}
\sphinxAtStartPar
Bases: \sphinxcode{\sphinxupquote{object}}

\sphinxAtStartPar
A class to manage and update the database.

\sphinxAtStartPar
Can get used to update all the stations and parameters at once.

\sphinxAtStartPar
This class is only working with super user privileges.
\index{\_\_init\_\_() (weatherDB.broker.Broker method)@\spxentry{\_\_init\_\_()}\spxextra{weatherDB.broker.Broker method}}

\begin{fulllineitems}
\phantomsection\label{\detokenize{weatherDB:weatherDB.broker.Broker.__init__}}\pysiglinewithargsret{\sphinxbfcode{\sphinxupquote{\_\_init\_\_}}}{}{}
\end{fulllineitems}

\index{fillup() (weatherDB.broker.Broker method)@\spxentry{fillup()}\spxextra{weatherDB.broker.Broker method}}

\begin{fulllineitems}
\phantomsection\label{\detokenize{weatherDB:weatherDB.broker.Broker.fillup}}\pysiglinewithargsret{\sphinxbfcode{\sphinxupquote{fillup}}}{\emph{\DUrole{n}{paras}\DUrole{o}{=}\DUrole{default_value}{{[}\textquotesingle{}n\textquotesingle{}, \textquotesingle{}t\textquotesingle{}, \textquotesingle{}et\textquotesingle{}{]}}}}{}
\sphinxAtStartPar
Fillup the timeseries.
\begin{quote}\begin{description}
\item[{Parameters}] \leavevmode
\sphinxAtStartPar
\sphinxstyleliteralstrong{\sphinxupquote{paras}} (\sphinxstyleliteralemphasis{\sphinxupquote{list of str}}\sphinxstyleliteralemphasis{\sphinxupquote{, }}\sphinxstyleliteralemphasis{\sphinxupquote{optional}}) \textendash{} The parameters for which to do the actions.
Can be one, some or all of {[}“n\_d”, “n”, “t”, “et”{]}.
The default is {[}“n\_d”, “n”, “t”, “et”{]}.

\end{description}\end{quote}

\end{fulllineitems}

\index{initiate\_db() (weatherDB.broker.Broker method)@\spxentry{initiate\_db()}\spxextra{weatherDB.broker.Broker method}}

\begin{fulllineitems}
\phantomsection\label{\detokenize{weatherDB:weatherDB.broker.Broker.initiate_db}}\pysiglinewithargsret{\sphinxbfcode{\sphinxupquote{initiate\_db}}}{}{}
\sphinxAtStartPar
Initiate the Database.

\sphinxAtStartPar
Downloads all the data from the CDC server for the first time.
Updates the multi\sphinxhyphen{}annual data and the richter\sphinxhyphen{}class for all the stations.
Quality checks and fills up the timeseries.

\end{fulllineitems}

\index{last\_imp\_corr() (weatherDB.broker.Broker method)@\spxentry{last\_imp\_corr()}\spxextra{weatherDB.broker.Broker method}}

\begin{fulllineitems}
\phantomsection\label{\detokenize{weatherDB:weatherDB.broker.Broker.last_imp_corr}}\pysiglinewithargsret{\sphinxbfcode{\sphinxupquote{last\_imp\_corr}}}{}{}
\sphinxAtStartPar
Richter correct the last imported precipitation data.

\end{fulllineitems}

\index{last\_imp\_fillup() (weatherDB.broker.Broker method)@\spxentry{last\_imp\_fillup()}\spxextra{weatherDB.broker.Broker method}}

\begin{fulllineitems}
\phantomsection\label{\detokenize{weatherDB:weatherDB.broker.Broker.last_imp_fillup}}\pysiglinewithargsret{\sphinxbfcode{\sphinxupquote{last\_imp\_fillup}}}{\emph{\DUrole{n}{paras}\DUrole{o}{=}\DUrole{default_value}{{[}\textquotesingle{}n\textquotesingle{}, \textquotesingle{}t\textquotesingle{}, \textquotesingle{}et\textquotesingle{}{]}}}}{}
\sphinxAtStartPar
Fillup the last imported data.
\begin{quote}\begin{description}
\item[{Parameters}] \leavevmode
\sphinxAtStartPar
\sphinxstyleliteralstrong{\sphinxupquote{paras}} (\sphinxstyleliteralemphasis{\sphinxupquote{list of str}}\sphinxstyleliteralemphasis{\sphinxupquote{, }}\sphinxstyleliteralemphasis{\sphinxupquote{optional}}) \textendash{} The parameters for which to do the actions.
Can be one, some or all of {[}“n\_d”, “n”, “t”, “et”{]}.
The default is {[}“n\_d”, “n”, “t”, “et”{]}.

\end{description}\end{quote}

\end{fulllineitems}

\index{last\_imp\_quality\_check() (weatherDB.broker.Broker method)@\spxentry{last\_imp\_quality\_check()}\spxextra{weatherDB.broker.Broker method}}

\begin{fulllineitems}
\phantomsection\label{\detokenize{weatherDB:weatherDB.broker.Broker.last_imp_quality_check}}\pysiglinewithargsret{\sphinxbfcode{\sphinxupquote{last\_imp\_quality\_check}}}{\emph{\DUrole{n}{paras}\DUrole{o}{=}\DUrole{default_value}{{[}\textquotesingle{}n\textquotesingle{}, \textquotesingle{}t\textquotesingle{}, \textquotesingle{}et\textquotesingle{}{]}}}, \emph{\DUrole{n}{with\_fillup\_nd}\DUrole{o}{=}\DUrole{default_value}{True}}}{}
\sphinxAtStartPar
Quality check the last imported data.

\sphinxAtStartPar
Also fills up the daily precipitation data if the 10 minute precipitation data should get quality checked.
\begin{quote}\begin{description}
\item[{Parameters}] \leavevmode\begin{itemize}
\item {} 
\sphinxAtStartPar
\sphinxstyleliteralstrong{\sphinxupquote{paras}} (\sphinxstyleliteralemphasis{\sphinxupquote{list of str}}\sphinxstyleliteralemphasis{\sphinxupquote{, }}\sphinxstyleliteralemphasis{\sphinxupquote{optional}}) \textendash{} The parameters for which to do the actions.
Can be one, some or all of {[}“n”, “t”, “et”{]}.
The default is {[}“n”, “t”, “et”{]}.

\item {} 
\sphinxAtStartPar
\sphinxstyleliteralstrong{\sphinxupquote{with\_fillup\_nd}} (\sphinxstyleliteralemphasis{\sphinxupquote{bool}}\sphinxstyleliteralemphasis{\sphinxupquote{, }}\sphinxstyleliteralemphasis{\sphinxupquote{optional}}) \textendash{} Should the daily precipitation data get filled up if the 10 minute precipitation data gets quality checked.
The default is True.

\end{itemize}

\end{description}\end{quote}

\end{fulllineitems}

\index{quality\_check() (weatherDB.broker.Broker method)@\spxentry{quality\_check()}\spxextra{weatherDB.broker.Broker method}}

\begin{fulllineitems}
\phantomsection\label{\detokenize{weatherDB:weatherDB.broker.Broker.quality_check}}\pysiglinewithargsret{\sphinxbfcode{\sphinxupquote{quality\_check}}}{\emph{\DUrole{n}{paras}\DUrole{o}{=}\DUrole{default_value}{{[}\textquotesingle{}n\textquotesingle{}, \textquotesingle{}t\textquotesingle{}, \textquotesingle{}et\textquotesingle{}{]}}}, \emph{\DUrole{n}{with\_fillup\_nd}\DUrole{o}{=}\DUrole{default_value}{True}}}{}
\sphinxAtStartPar
Do the quality check on the stations raw data.
\begin{quote}\begin{description}
\item[{Parameters}] \leavevmode\begin{itemize}
\item {} 
\sphinxAtStartPar
\sphinxstyleliteralstrong{\sphinxupquote{paras}} (\sphinxstyleliteralemphasis{\sphinxupquote{list of str}}\sphinxstyleliteralemphasis{\sphinxupquote{, }}\sphinxstyleliteralemphasis{\sphinxupquote{optional}}) \textendash{} The parameters for which to do the actions.
Can be one, some or all of {[}“n”, “t”, “et”{]}.
The default is {[}“n”, “t”, “et”{]}.

\item {} 
\sphinxAtStartPar
\sphinxstyleliteralstrong{\sphinxupquote{with\_fillup\_nd}} (\sphinxstyleliteralemphasis{\sphinxupquote{bool}}\sphinxstyleliteralemphasis{\sphinxupquote{, }}\sphinxstyleliteralemphasis{\sphinxupquote{optional}}) \textendash{} Should the daily precipitation data get filled up if the 10 minute precipitation data gets quality checked.
The default is True.

\end{itemize}

\end{description}\end{quote}

\end{fulllineitems}

\index{richter\_correct() (weatherDB.broker.Broker method)@\spxentry{richter\_correct()}\spxextra{weatherDB.broker.Broker method}}

\begin{fulllineitems}
\phantomsection\label{\detokenize{weatherDB:weatherDB.broker.Broker.richter_correct}}\pysiglinewithargsret{\sphinxbfcode{\sphinxupquote{richter\_correct}}}{}{}
\sphinxAtStartPar
Richter correct all of the precipitation data.

\end{fulllineitems}

\index{update\_db() (weatherDB.broker.Broker method)@\spxentry{update\_db()}\spxextra{weatherDB.broker.Broker method}}

\begin{fulllineitems}
\phantomsection\label{\detokenize{weatherDB:weatherDB.broker.Broker.update_db}}\pysiglinewithargsret{\sphinxbfcode{\sphinxupquote{update\_db}}}{\emph{\DUrole{n}{paras}\DUrole{o}{=}\DUrole{default_value}{{[}\textquotesingle{}n\_d\textquotesingle{}, \textquotesingle{}n\textquotesingle{}, \textquotesingle{}t\textquotesingle{}, \textquotesingle{}et\textquotesingle{}{]}}}}{}
\sphinxAtStartPar
The regular Update of the database.

\sphinxAtStartPar
Downloads new data.
Quality checks the newly imported data.
Fills up the newly imported data.
\begin{quote}\begin{description}
\item[{Parameters}] \leavevmode
\sphinxAtStartPar
\sphinxstyleliteralstrong{\sphinxupquote{paras}} (\sphinxstyleliteralemphasis{\sphinxupquote{list of str}}\sphinxstyleliteralemphasis{\sphinxupquote{, }}\sphinxstyleliteralemphasis{\sphinxupquote{optional}}) \textendash{} The parameters for which to do the actions.
Can be one, some or all of {[}“n\_d”, “n”, “t”, “et”{]}.
The default is {[}“n\_d”, “n”, “t”, “et”{]}.

\end{description}\end{quote}

\end{fulllineitems}

\index{update\_ma() (weatherDB.broker.Broker method)@\spxentry{update\_ma()}\spxextra{weatherDB.broker.Broker method}}

\begin{fulllineitems}
\phantomsection\label{\detokenize{weatherDB:weatherDB.broker.Broker.update_ma}}\pysiglinewithargsret{\sphinxbfcode{\sphinxupquote{update\_ma}}}{\emph{\DUrole{n}{paras}\DUrole{o}{=}\DUrole{default_value}{{[}\textquotesingle{}n\_d\textquotesingle{}, \textquotesingle{}n\textquotesingle{}, \textquotesingle{}t\textquotesingle{}, \textquotesingle{}et\textquotesingle{}{]}}}}{}
\sphinxAtStartPar
Update the multi\sphinxhyphen{}annual data from raster to table.
\begin{quote}\begin{description}
\item[{Parameters}] \leavevmode
\sphinxAtStartPar
\sphinxstyleliteralstrong{\sphinxupquote{paras}} (\sphinxstyleliteralemphasis{\sphinxupquote{list of str}}\sphinxstyleliteralemphasis{\sphinxupquote{, }}\sphinxstyleliteralemphasis{\sphinxupquote{optional}}) \textendash{} The parameters for which to do the actions.
Can be one, some or all of {[}“n\_d”, “n”, “t”, “et”{]}.
The default is {[}“n\_d”, “n”, “t”, “et”{]}.

\end{description}\end{quote}

\end{fulllineitems}

\index{update\_meta() (weatherDB.broker.Broker method)@\spxentry{update\_meta()}\spxextra{weatherDB.broker.Broker method}}

\begin{fulllineitems}
\phantomsection\label{\detokenize{weatherDB:weatherDB.broker.Broker.update_meta}}\pysiglinewithargsret{\sphinxbfcode{\sphinxupquote{update\_meta}}}{\emph{\DUrole{n}{paras}\DUrole{o}{=}\DUrole{default_value}{{[}\textquotesingle{}n\_d\textquotesingle{}, \textquotesingle{}n\textquotesingle{}, \textquotesingle{}t\textquotesingle{}, \textquotesingle{}et\textquotesingle{}{]}}}}{}
\sphinxAtStartPar
Update the meta file from the CDC Server to the Database.
\begin{quote}\begin{description}
\item[{Parameters}] \leavevmode
\sphinxAtStartPar
\sphinxstyleliteralstrong{\sphinxupquote{paras}} (\sphinxstyleliteralemphasis{\sphinxupquote{list of str}}\sphinxstyleliteralemphasis{\sphinxupquote{, }}\sphinxstyleliteralemphasis{\sphinxupquote{optional}}) \textendash{} The parameters for which to do the actions.
Can be one, some or all of {[}“n\_d”, “n”, “t”, “et”{]}.
The default is {[}“n\_d”, “n”, “t”, “et”{]}.

\end{description}\end{quote}

\end{fulllineitems}

\index{update\_period\_meta() (weatherDB.broker.Broker method)@\spxentry{update\_period\_meta()}\spxextra{weatherDB.broker.Broker method}}

\begin{fulllineitems}
\phantomsection\label{\detokenize{weatherDB:weatherDB.broker.Broker.update_period_meta}}\pysiglinewithargsret{\sphinxbfcode{\sphinxupquote{update\_period\_meta}}}{\emph{\DUrole{n}{paras}\DUrole{o}{=}\DUrole{default_value}{{[}\textquotesingle{}n\_d\textquotesingle{}, \textquotesingle{}n\textquotesingle{}, \textquotesingle{}t\textquotesingle{}, \textquotesingle{}et\textquotesingle{}{]}}}}{}
\sphinxAtStartPar
Update the periods in the meta table.
\begin{quote}\begin{description}
\item[{Parameters}] \leavevmode
\sphinxAtStartPar
\sphinxstyleliteralstrong{\sphinxupquote{paras}} (\sphinxstyleliteralemphasis{\sphinxupquote{list of str}}\sphinxstyleliteralemphasis{\sphinxupquote{, }}\sphinxstyleliteralemphasis{\sphinxupquote{optional}}) \textendash{} The parameters for which to do the actions.
Can be one, some or all of {[}“n\_d”, “n”, “t”, “et”{]}.
The default is {[}“n\_d”, “n”, “t”, “et”{]}.

\end{description}\end{quote}

\end{fulllineitems}

\index{update\_raw() (weatherDB.broker.Broker method)@\spxentry{update\_raw()}\spxextra{weatherDB.broker.Broker method}}

\begin{fulllineitems}
\phantomsection\label{\detokenize{weatherDB:weatherDB.broker.Broker.update_raw}}\pysiglinewithargsret{\sphinxbfcode{\sphinxupquote{update\_raw}}}{\emph{\DUrole{n}{only\_new}\DUrole{o}{=}\DUrole{default_value}{True}}, \emph{\DUrole{n}{paras}\DUrole{o}{=}\DUrole{default_value}{{[}\textquotesingle{}n\_d\textquotesingle{}, \textquotesingle{}n\textquotesingle{}, \textquotesingle{}t\textquotesingle{}, \textquotesingle{}et\textquotesingle{}{]}}}}{}
\sphinxAtStartPar
Update the raw data from the DWD\sphinxhyphen{}CDC server to the database.
\begin{quote}\begin{description}
\item[{Parameters}] \leavevmode\begin{itemize}
\item {} 
\sphinxAtStartPar
\sphinxstyleliteralstrong{\sphinxupquote{only\_new}} (\sphinxstyleliteralemphasis{\sphinxupquote{bool}}\sphinxstyleliteralemphasis{\sphinxupquote{, }}\sphinxstyleliteralemphasis{\sphinxupquote{optional}}) \textendash{} Get only the files that are not yet in the database?
If False all the available files are loaded again.
The default is True.

\item {} 
\sphinxAtStartPar
\sphinxstyleliteralstrong{\sphinxupquote{paras}} (\sphinxstyleliteralemphasis{\sphinxupquote{list of str}}\sphinxstyleliteralemphasis{\sphinxupquote{, }}\sphinxstyleliteralemphasis{\sphinxupquote{optional}}) \textendash{} The parameters for which to do the actions.
Can be one, some or all of {[}“n\_d”, “n”, “t”, “et”{]}.
The default is {[}“n\_d”, “n”, “t”, “et”{]}.

\end{itemize}

\end{description}\end{quote}

\end{fulllineitems}


\end{fulllineitems}



\subsection{weatherDB.station module}
\label{\detokenize{weatherDB:module-weatherDB.station}}\label{\detokenize{weatherDB:weatherdb-station-module}}\index{module@\spxentry{module}!weatherDB.station@\spxentry{weatherDB.station}}\index{weatherDB.station@\spxentry{weatherDB.station}!module@\spxentry{module}}\index{EvapotranspirationStation (class in weatherDB.station)@\spxentry{EvapotranspirationStation}\spxextra{class in weatherDB.station}}

\begin{fulllineitems}
\phantomsection\label{\detokenize{weatherDB:weatherDB.station.EvapotranspirationStation}}\pysiglinewithargsret{\sphinxbfcode{\sphinxupquote{class\DUrole{w}{  }}}\sphinxcode{\sphinxupquote{weatherDB.station.}}\sphinxbfcode{\sphinxupquote{EvapotranspirationStation}}}{\emph{\DUrole{n}{id}}}{}
\sphinxAtStartPar
Bases: {\hyperref[\detokenize{weatherDB:weatherDB.station.StationTETBase}]{\sphinxcrossref{\sphinxcode{\sphinxupquote{weatherDB.station.StationTETBase}}}}}
\index{\_\_init\_\_() (weatherDB.station.EvapotranspirationStation method)@\spxentry{\_\_init\_\_()}\spxextra{weatherDB.station.EvapotranspirationStation method}}

\begin{fulllineitems}
\phantomsection\label{\detokenize{weatherDB:weatherDB.station.EvapotranspirationStation.__init__}}\pysiglinewithargsret{\sphinxbfcode{\sphinxupquote{\_\_init\_\_}}}{\emph{\DUrole{n}{id}}}{}
\end{fulllineitems}

\index{get\_adj() (weatherDB.station.EvapotranspirationStation method)@\spxentry{get\_adj()}\spxextra{weatherDB.station.EvapotranspirationStation method}}

\begin{fulllineitems}
\phantomsection\label{\detokenize{weatherDB:weatherDB.station.EvapotranspirationStation.get_adj}}\pysiglinewithargsret{\sphinxbfcode{\sphinxupquote{get\_adj}}}{\emph{\DUrole{n}{period}\DUrole{o}{=}\DUrole{default_value}{(None, None)}}}{}
\sphinxAtStartPar
Get the adjusted timeserie.

\sphinxAtStartPar
The timeserie is adjusted to the multi annual mean.
So the overall mean of the given period will be the same as the multi annual mean.
\begin{quote}\begin{description}
\item[{Parameters}] \leavevmode\begin{itemize}
\item {} 
\sphinxAtStartPar
\sphinxstyleliteralstrong{\sphinxupquote{period}} ({\hyperref[\detokenize{weatherDB.lib:weatherDB.lib.utils.TimestampPeriod}]{\sphinxcrossref{\sphinxstyleliteralemphasis{\sphinxupquote{TimestampPeriod}}}}}\sphinxstyleliteralemphasis{\sphinxupquote{ or }}\sphinxstyleliteralemphasis{\sphinxupquote{(}}\sphinxstyleliteralemphasis{\sphinxupquote{tuple}}\sphinxstyleliteralemphasis{\sphinxupquote{ or }}\sphinxstyleliteralemphasis{\sphinxupquote{list of datetime.datetime}}\sphinxstyleliteralemphasis{\sphinxupquote{ or }}\sphinxstyleliteralemphasis{\sphinxupquote{None}}\sphinxstyleliteralemphasis{\sphinxupquote{)}}\sphinxstyleliteralemphasis{\sphinxupquote{, }}\sphinxstyleliteralemphasis{\sphinxupquote{optional}}) \textendash{} The minimum and maximum Timestamp for which to get the timeseries.
If None is given, the maximum or minimal possible Timestamp is taken.
The default is (None, None).

\item {} 
\sphinxAtStartPar
\sphinxstyleliteralstrong{\sphinxupquote{agg\_to}} (\sphinxstyleliteralemphasis{\sphinxupquote{str}}\sphinxstyleliteralemphasis{\sphinxupquote{ or }}\sphinxstyleliteralemphasis{\sphinxupquote{None}}\sphinxstyleliteralemphasis{\sphinxupquote{, }}\sphinxstyleliteralemphasis{\sphinxupquote{optional}}) \textendash{} Aggregate to a given timespan.
Can be anything smaller than the maximum timespan of the saved data.
If a Timeperiod smaller than the saved data is given, than the maximum possible timeperiod is returned.
For T and ET it can be “month”, “year”.
For N it can also be “hour”.
If None than the maximum timeperiod is taken.
The default is None.

\end{itemize}

\item[{Returns}] \leavevmode
\sphinxAtStartPar
A timeserie with the adjusted data.

\item[{Return type}] \leavevmode
\sphinxAtStartPar
pandas.DataFrame

\end{description}\end{quote}

\end{fulllineitems}


\end{fulllineitems}

\index{GroupStation (class in weatherDB.station)@\spxentry{GroupStation}\spxextra{class in weatherDB.station}}

\begin{fulllineitems}
\phantomsection\label{\detokenize{weatherDB:weatherDB.station.GroupStation}}\pysiglinewithargsret{\sphinxbfcode{\sphinxupquote{class\DUrole{w}{  }}}\sphinxcode{\sphinxupquote{weatherDB.station.}}\sphinxbfcode{\sphinxupquote{GroupStation}}}{\emph{\DUrole{n}{id}}, \emph{\DUrole{n}{error\_if\_missing}\DUrole{o}{=}\DUrole{default_value}{True}}}{}
\sphinxAtStartPar
Bases: \sphinxcode{\sphinxupquote{object}}

\sphinxAtStartPar
A class to group all possible parameters of one station.
\index{\_\_init\_\_() (weatherDB.station.GroupStation method)@\spxentry{\_\_init\_\_()}\spxextra{weatherDB.station.GroupStation method}}

\begin{fulllineitems}
\phantomsection\label{\detokenize{weatherDB:weatherDB.station.GroupStation.__init__}}\pysiglinewithargsret{\sphinxbfcode{\sphinxupquote{\_\_init\_\_}}}{\emph{\DUrole{n}{id}}, \emph{\DUrole{n}{error\_if\_missing}\DUrole{o}{=}\DUrole{default_value}{True}}}{}
\end{fulllineitems}

\index{create\_roger\_ts() (weatherDB.station.GroupStation method)@\spxentry{create\_roger\_ts()}\spxextra{weatherDB.station.GroupStation method}}

\begin{fulllineitems}
\phantomsection\label{\detokenize{weatherDB:weatherDB.station.GroupStation.create_roger_ts}}\pysiglinewithargsret{\sphinxbfcode{\sphinxupquote{create\_roger\_ts}}}{\emph{\DUrole{n}{dir}}, \emph{\DUrole{n}{period}\DUrole{o}{=}\DUrole{default_value}{(None, None)}}, \emph{\DUrole{n}{kind}\DUrole{o}{=}\DUrole{default_value}{\textquotesingle{}best\textquotesingle{}}}}{}
\sphinxAtStartPar
Create the timeserie files for RoGeR.
\begin{quote}\begin{description}
\item[{Parameters}] \leavevmode\begin{itemize}
\item {} 
\sphinxAtStartPar
\sphinxstyleliteralstrong{\sphinxupquote{dir}} (\sphinxstyleliteralemphasis{\sphinxupquote{pathlib like object}}) \textendash{} The directory to store the timeseries in.

\item {} 
\sphinxAtStartPar
\sphinxstyleliteralstrong{\sphinxupquote{period}} (\sphinxstyleliteralemphasis{\sphinxupquote{TimestampPeriod like object}}\sphinxstyleliteralemphasis{\sphinxupquote{, }}\sphinxstyleliteralemphasis{\sphinxupquote{optional}}) \textendash{} The period for which to get the timeseries.
If (None, None) is entered, then the maximal possible period is computed.
The default is (None, None)

\item {} 
\sphinxAtStartPar
\sphinxstyleliteralstrong{\sphinxupquote{kind}} (\sphinxstyleliteralemphasis{\sphinxupquote{str}}) \textendash{} The data kind to look for filled period.
Must be a column in the timeseries DB.
Must be one of “raw”, “qc”, “filled”, “adj”.
If “best” is given, then depending on the parameter of the station the best kind is selected.
For Precipitation this is “corr” and for the other this is “filled”.
For the precipitation also “qn” and “corr” are valid.

\end{itemize}

\item[{Raises}] \leavevmode
\sphinxAtStartPar
\sphinxstyleliteralstrong{\sphinxupquote{Warning}} \textendash{} If there are NAs in the timeseries or the period got changed.

\end{description}\end{quote}

\end{fulllineitems}

\index{get\_df() (weatherDB.station.GroupStation method)@\spxentry{get\_df()}\spxextra{weatherDB.station.GroupStation method}}

\begin{fulllineitems}
\phantomsection\label{\detokenize{weatherDB:weatherDB.station.GroupStation.get_df}}\pysiglinewithargsret{\sphinxbfcode{\sphinxupquote{get\_df}}}{\emph{\DUrole{n}{period}\DUrole{o}{=}\DUrole{default_value}{(None, None)}}, \emph{\DUrole{n}{kind}\DUrole{o}{=}\DUrole{default_value}{\textquotesingle{}best\textquotesingle{}}}, \emph{\DUrole{n}{paras}\DUrole{o}{=}\DUrole{default_value}{\textquotesingle{}all\textquotesingle{}}}}{}
\sphinxAtStartPar
Get a DataFrame with the corresponding data.
\begin{quote}\begin{description}
\item[{Parameters}] \leavevmode\begin{itemize}
\item {} 
\sphinxAtStartPar
\sphinxstyleliteralstrong{\sphinxupquote{period}} ({\hyperref[\detokenize{weatherDB.lib:weatherDB.lib.utils.TimestampPeriod}]{\sphinxcrossref{\sphinxstyleliteralemphasis{\sphinxupquote{TimestampPeriod}}}}}\sphinxstyleliteralemphasis{\sphinxupquote{ or }}\sphinxstyleliteralemphasis{\sphinxupquote{(}}\sphinxstyleliteralemphasis{\sphinxupquote{tuple}}\sphinxstyleliteralemphasis{\sphinxupquote{ or }}\sphinxstyleliteralemphasis{\sphinxupquote{list of datetime.datetime}}\sphinxstyleliteralemphasis{\sphinxupquote{ or }}\sphinxstyleliteralemphasis{\sphinxupquote{None}}\sphinxstyleliteralemphasis{\sphinxupquote{)}}\sphinxstyleliteralemphasis{\sphinxupquote{, }}\sphinxstyleliteralemphasis{\sphinxupquote{optional}}) \textendash{} The minimum and maximum Timestamp for which to get the timeseries.
If None is given, the maximum or minimal possible Timestamp is taken.
The default is (None, None).

\item {} 
\sphinxAtStartPar
\sphinxstyleliteralstrong{\sphinxupquote{kind}} (\sphinxstyleliteralemphasis{\sphinxupquote{str}}) \textendash{} The data kind to look for filled period.
Must be a column in the timeseries DB.
Must be one of “raw”, “qc”, “filled”, “adj”.
If “best” is given, then depending on the parameter of the station the best kind is selected.
For Precipitation this is “corr” and for the other this is “filled”.
For the precipitation also “qn” and “corr” are valid.

\end{itemize}

\item[{Returns}] \leavevmode
\sphinxAtStartPar
A DataFrame with the timeseries for this station and the given period.

\item[{Return type}] \leavevmode
\sphinxAtStartPar
pd.Dataframe

\end{description}\end{quote}

\end{fulllineitems}

\index{get\_filled\_period() (weatherDB.station.GroupStation method)@\spxentry{get\_filled\_period()}\spxextra{weatherDB.station.GroupStation method}}

\begin{fulllineitems}
\phantomsection\label{\detokenize{weatherDB:weatherDB.station.GroupStation.get_filled_period}}\pysiglinewithargsret{\sphinxbfcode{\sphinxupquote{get\_filled\_period}}}{\emph{\DUrole{n}{kind}\DUrole{o}{=}\DUrole{default_value}{\textquotesingle{}best\textquotesingle{}}}, \emph{\DUrole{n}{from\_meta}\DUrole{o}{=}\DUrole{default_value}{True}}}{}
\sphinxAtStartPar
Get the combined filled period for all 3 stations.

\sphinxAtStartPar
This is the maximum possible timerange for these stations.
\begin{quote}\begin{description}
\item[{Parameters}] \leavevmode\begin{itemize}
\item {} 
\sphinxAtStartPar
\sphinxstyleliteralstrong{\sphinxupquote{kind}} (\sphinxstyleliteralemphasis{\sphinxupquote{str}}) \textendash{} The data kind to look for filled period.
Must be a column in the timeseries DB.
Must be one of “raw”, “qc”, “filled”, “adj”.
If “best” is given, then depending on the parameter of the station the best kind is selected.
For Precipitation this is “corr” and for the other this is “filled”.
For the precipitation also “qn” and “corr” are valid.

\item {} 
\sphinxAtStartPar
\sphinxstyleliteralstrong{\sphinxupquote{from\_meta}} (\sphinxstyleliteralemphasis{\sphinxupquote{bool}}\sphinxstyleliteralemphasis{\sphinxupquote{, }}\sphinxstyleliteralemphasis{\sphinxupquote{optional}}) \textendash{} Should the period be from the meta table?
If False: the period is returned from the timeserie. In this case this function is only a wrapper for .get\_period\_meta.
The default is True.

\end{itemize}

\item[{Returns}] \leavevmode
\sphinxAtStartPar
The maximum filled period for the 3 parameters for this station.

\item[{Return type}] \leavevmode
\sphinxAtStartPar
{\hyperref[\detokenize{weatherDB.lib:weatherDB.lib.utils.TimestampPeriod}]{\sphinxcrossref{TimestampPeriod}}}

\end{description}\end{quote}

\end{fulllineitems}

\index{get\_geom() (weatherDB.station.GroupStation method)@\spxentry{get\_geom()}\spxextra{weatherDB.station.GroupStation method}}

\begin{fulllineitems}
\phantomsection\label{\detokenize{weatherDB:weatherDB.station.GroupStation.get_geom}}\pysiglinewithargsret{\sphinxbfcode{\sphinxupquote{get\_geom}}}{}{}
\end{fulllineitems}

\index{get\_name() (weatherDB.station.GroupStation method)@\spxentry{get\_name()}\spxextra{weatherDB.station.GroupStation method}}

\begin{fulllineitems}
\phantomsection\label{\detokenize{weatherDB:weatherDB.station.GroupStation.get_name}}\pysiglinewithargsret{\sphinxbfcode{\sphinxupquote{get\_name}}}{}{}
\end{fulllineitems}

\index{get\_possible\_paras() (weatherDB.station.GroupStation method)@\spxentry{get\_possible\_paras()}\spxextra{weatherDB.station.GroupStation method}}

\begin{fulllineitems}
\phantomsection\label{\detokenize{weatherDB:weatherDB.station.GroupStation.get_possible_paras}}\pysiglinewithargsret{\sphinxbfcode{\sphinxupquote{get\_possible\_paras}}}{\emph{\DUrole{n}{short}\DUrole{o}{=}\DUrole{default_value}{False}}}{}
\sphinxAtStartPar
Get the possible parameters for this station.
\begin{quote}\begin{description}
\item[{Parameters}] \leavevmode
\sphinxAtStartPar
\sphinxstyleliteralstrong{\sphinxupquote{short}} (\sphinxstyleliteralemphasis{\sphinxupquote{bool}}\sphinxstyleliteralemphasis{\sphinxupquote{, }}\sphinxstyleliteralemphasis{\sphinxupquote{optional}}) \textendash{} Should the short name of the parameters be returned.
The default is “long”.

\item[{Returns}] \leavevmode
\sphinxAtStartPar
A list of the long parameter names that are possible for this station to get.

\item[{Return type}] \leavevmode
\sphinxAtStartPar
list of str

\end{description}\end{quote}

\end{fulllineitems}


\end{fulllineitems}

\index{PrecipitationDailyStation (class in weatherDB.station)@\spxentry{PrecipitationDailyStation}\spxextra{class in weatherDB.station}}

\begin{fulllineitems}
\phantomsection\label{\detokenize{weatherDB:weatherDB.station.PrecipitationDailyStation}}\pysiglinewithargsret{\sphinxbfcode{\sphinxupquote{class\DUrole{w}{  }}}\sphinxcode{\sphinxupquote{weatherDB.station.}}\sphinxbfcode{\sphinxupquote{PrecipitationDailyStation}}}{\emph{\DUrole{n}{id}}}{}
\sphinxAtStartPar
Bases: {\hyperref[\detokenize{weatherDB:weatherDB.station.StationNBase}]{\sphinxcrossref{\sphinxcode{\sphinxupquote{weatherDB.station.StationNBase}}}}}
\index{\_\_init\_\_() (weatherDB.station.PrecipitationDailyStation method)@\spxentry{\_\_init\_\_()}\spxextra{weatherDB.station.PrecipitationDailyStation method}}

\begin{fulllineitems}
\phantomsection\label{\detokenize{weatherDB:weatherDB.station.PrecipitationDailyStation.__init__}}\pysiglinewithargsret{\sphinxbfcode{\sphinxupquote{\_\_init\_\_}}}{\emph{\DUrole{n}{id}}}{}
\end{fulllineitems}

\index{get\_adj (weatherDB.station.PrecipitationDailyStation property)@\spxentry{get\_adj}\spxextra{weatherDB.station.PrecipitationDailyStation property}}

\begin{fulllineitems}
\phantomsection\label{\detokenize{weatherDB:weatherDB.station.PrecipitationDailyStation.get_adj}}\pysigline{\sphinxbfcode{\sphinxupquote{property\DUrole{w}{  }}}\sphinxbfcode{\sphinxupquote{get\_adj}}}
\sphinxAtStartPar
(!) Disallowed inherited

\end{fulllineitems}

\index{get\_corr (weatherDB.station.PrecipitationDailyStation property)@\spxentry{get\_corr}\spxextra{weatherDB.station.PrecipitationDailyStation property}}

\begin{fulllineitems}
\phantomsection\label{\detokenize{weatherDB:weatherDB.station.PrecipitationDailyStation.get_corr}}\pysigline{\sphinxbfcode{\sphinxupquote{property\DUrole{w}{  }}}\sphinxbfcode{\sphinxupquote{get\_corr}}}
\sphinxAtStartPar
(!) Disallowed inherited

\end{fulllineitems}

\index{get\_qc (weatherDB.station.PrecipitationDailyStation property)@\spxentry{get\_qc}\spxextra{weatherDB.station.PrecipitationDailyStation property}}

\begin{fulllineitems}
\phantomsection\label{\detokenize{weatherDB:weatherDB.station.PrecipitationDailyStation.get_qc}}\pysigline{\sphinxbfcode{\sphinxupquote{property\DUrole{w}{  }}}\sphinxbfcode{\sphinxupquote{get\_qc}}}
\sphinxAtStartPar
(!) Disallowed inherited

\end{fulllineitems}

\index{last\_imp\_quality\_check (weatherDB.station.PrecipitationDailyStation property)@\spxentry{last\_imp\_quality\_check}\spxextra{weatherDB.station.PrecipitationDailyStation property}}

\begin{fulllineitems}
\phantomsection\label{\detokenize{weatherDB:weatherDB.station.PrecipitationDailyStation.last_imp_quality_check}}\pysigline{\sphinxbfcode{\sphinxupquote{property\DUrole{w}{  }}}\sphinxbfcode{\sphinxupquote{last\_imp\_quality\_check}}}
\sphinxAtStartPar
(!) Disallowed inherited

\end{fulllineitems}

\index{quality\_check (weatherDB.station.PrecipitationDailyStation property)@\spxentry{quality\_check}\spxextra{weatherDB.station.PrecipitationDailyStation property}}

\begin{fulllineitems}
\phantomsection\label{\detokenize{weatherDB:weatherDB.station.PrecipitationDailyStation.quality_check}}\pysigline{\sphinxbfcode{\sphinxupquote{property\DUrole{w}{  }}}\sphinxbfcode{\sphinxupquote{quality\_check}}}
\sphinxAtStartPar
(!) Disallowed inherited

\end{fulllineitems}


\end{fulllineitems}

\index{PrecipitationStation (class in weatherDB.station)@\spxentry{PrecipitationStation}\spxextra{class in weatherDB.station}}

\begin{fulllineitems}
\phantomsection\label{\detokenize{weatherDB:weatherDB.station.PrecipitationStation}}\pysiglinewithargsret{\sphinxbfcode{\sphinxupquote{class\DUrole{w}{  }}}\sphinxcode{\sphinxupquote{weatherDB.station.}}\sphinxbfcode{\sphinxupquote{PrecipitationStation}}}{\emph{\DUrole{n}{id}}}{}
\sphinxAtStartPar
Bases: {\hyperref[\detokenize{weatherDB:weatherDB.station.StationNBase}]{\sphinxcrossref{\sphinxcode{\sphinxupquote{weatherDB.station.StationNBase}}}}}
\index{\_\_init\_\_() (weatherDB.station.PrecipitationStation method)@\spxentry{\_\_init\_\_()}\spxextra{weatherDB.station.PrecipitationStation method}}

\begin{fulllineitems}
\phantomsection\label{\detokenize{weatherDB:weatherDB.station.PrecipitationStation.__init__}}\pysiglinewithargsret{\sphinxbfcode{\sphinxupquote{\_\_init\_\_}}}{\emph{\DUrole{n}{id}}}{}
\end{fulllineitems}

\index{corr() (weatherDB.station.PrecipitationStation method)@\spxentry{corr()}\spxextra{weatherDB.station.PrecipitationStation method}}

\begin{fulllineitems}
\phantomsection\label{\detokenize{weatherDB:weatherDB.station.PrecipitationStation.corr}}\pysiglinewithargsret{\sphinxbfcode{\sphinxupquote{corr}}}{\emph{\DUrole{n}{period}\DUrole{o}{=}\DUrole{default_value}{(None, None)}}}{}
\end{fulllineitems}

\index{fillup() (weatherDB.station.PrecipitationStation method)@\spxentry{fillup()}\spxextra{weatherDB.station.PrecipitationStation method}}

\begin{fulllineitems}
\phantomsection\label{\detokenize{weatherDB:weatherDB.station.PrecipitationStation.fillup}}\pysiglinewithargsret{\sphinxbfcode{\sphinxupquote{fillup}}}{\emph{\DUrole{n}{period}\DUrole{o}{=}\DUrole{default_value}{(None, None)}}}{}
\sphinxAtStartPar
Fill up missing data with measurements from nearby stations.

\end{fulllineitems}

\index{get\_corr() (weatherDB.station.PrecipitationStation method)@\spxentry{get\_corr()}\spxextra{weatherDB.station.PrecipitationStation method}}

\begin{fulllineitems}
\phantomsection\label{\detokenize{weatherDB:weatherDB.station.PrecipitationStation.get_corr}}\pysiglinewithargsret{\sphinxbfcode{\sphinxupquote{get\_corr}}}{\emph{\DUrole{n}{period}\DUrole{o}{=}\DUrole{default_value}{(None, None)}}}{}
\end{fulllineitems}

\index{get\_horizon() (weatherDB.station.PrecipitationStation method)@\spxentry{get\_horizon()}\spxextra{weatherDB.station.PrecipitationStation method}}

\begin{fulllineitems}
\phantomsection\label{\detokenize{weatherDB:weatherDB.station.PrecipitationStation.get_horizon}}\pysiglinewithargsret{\sphinxbfcode{\sphinxupquote{get\_horizon}}}{}{}
\sphinxAtStartPar
Get the value for the horizon angle. (Horizontabschirmung)

\sphinxAtStartPar
This value is defined by Richter (1995) as the mean horizon angle in the west direction as:
H’=0,15H(S\sphinxhyphen{}SW) +0,35H(SW\sphinxhyphen{}W) +0,35H(W\sphinxhyphen{}NW) +0, 15H(NW\sphinxhyphen{}N)
\begin{quote}\begin{description}
\item[{Returns}] \leavevmode
\sphinxAtStartPar
The mean western horizon angle

\item[{Return type}] \leavevmode
\sphinxAtStartPar
float or None

\end{description}\end{quote}

\end{fulllineitems}

\index{get\_qn() (weatherDB.station.PrecipitationStation method)@\spxentry{get\_qn()}\spxextra{weatherDB.station.PrecipitationStation method}}

\begin{fulllineitems}
\phantomsection\label{\detokenize{weatherDB:weatherDB.station.PrecipitationStation.get_qn}}\pysiglinewithargsret{\sphinxbfcode{\sphinxupquote{get\_qn}}}{\emph{\DUrole{n}{period}\DUrole{o}{=}\DUrole{default_value}{(None, None)}}}{}
\end{fulllineitems}

\index{get\_richter\_class() (weatherDB.station.PrecipitationStation method)@\spxentry{get\_richter\_class()}\spxextra{weatherDB.station.PrecipitationStation method}}

\begin{fulllineitems}
\phantomsection\label{\detokenize{weatherDB:weatherDB.station.PrecipitationStation.get_richter_class}}\pysiglinewithargsret{\sphinxbfcode{\sphinxupquote{get\_richter\_class}}}{\emph{\DUrole{n}{update\_if\_fails}\DUrole{o}{=}\DUrole{default_value}{True}}}{}
\sphinxAtStartPar
Get the richter class for this station.

\sphinxAtStartPar
Provide the data from the meta table.
\begin{quote}\begin{description}
\item[{Parameters}] \leavevmode
\sphinxAtStartPar
\sphinxstyleliteralstrong{\sphinxupquote{update\_if\_fails}} (\sphinxstyleliteralemphasis{\sphinxupquote{bool}}\sphinxstyleliteralemphasis{\sphinxupquote{, }}\sphinxstyleliteralemphasis{\sphinxupquote{optional}}) \textendash{} Should the richter class get updatet if no exposition class is found in the meta table?
If False and no exposition class was found None is returned.
The default is True.

\item[{Returns}] \leavevmode
\sphinxAtStartPar
The corresponding richter exposition class.

\item[{Return type}] \leavevmode
\sphinxAtStartPar
string

\end{description}\end{quote}

\end{fulllineitems}

\index{last\_imp\_corr() (weatherDB.station.PrecipitationStation method)@\spxentry{last\_imp\_corr()}\spxextra{weatherDB.station.PrecipitationStation method}}

\begin{fulllineitems}
\phantomsection\label{\detokenize{weatherDB:weatherDB.station.PrecipitationStation.last_imp_corr}}\pysiglinewithargsret{\sphinxbfcode{\sphinxupquote{last\_imp\_corr}}}{\emph{\DUrole{n}{\_last\_imp\_period}\DUrole{o}{=}\DUrole{default_value}{None}}}{}
\sphinxAtStartPar
A wrapper for last\_imp\_richter\_correct().

\end{fulllineitems}

\index{last\_imp\_richter\_correct() (weatherDB.station.PrecipitationStation method)@\spxentry{last\_imp\_richter\_correct()}\spxextra{weatherDB.station.PrecipitationStation method}}

\begin{fulllineitems}
\phantomsection\label{\detokenize{weatherDB:weatherDB.station.PrecipitationStation.last_imp_richter_correct}}\pysiglinewithargsret{\sphinxbfcode{\sphinxupquote{last\_imp\_richter\_correct}}}{\emph{\DUrole{n}{\_last\_imp\_period}\DUrole{o}{=}\DUrole{default_value}{None}}}{}
\sphinxAtStartPar
Do the richter correction of the last import.
\begin{quote}\begin{description}
\item[{Parameters}] \leavevmode
\sphinxAtStartPar
\sphinxstyleliteralstrong{\sphinxupquote{\_last\_imp\_period}} (\sphinxstyleliteralemphasis{\sphinxupquote{\_type\_}}\sphinxstyleliteralemphasis{\sphinxupquote{, }}\sphinxstyleliteralemphasis{\sphinxupquote{optional}}) \textendash{} Give the overall period of the last import.
This is only for intern use of the stationsN methode to not compute over and over again the period.
The default is None.

\end{description}\end{quote}

\end{fulllineitems}

\index{richter\_correct() (weatherDB.station.PrecipitationStation method)@\spxentry{richter\_correct()}\spxextra{weatherDB.station.PrecipitationStation method}}

\begin{fulllineitems}
\phantomsection\label{\detokenize{weatherDB:weatherDB.station.PrecipitationStation.richter_correct}}\pysiglinewithargsret{\sphinxbfcode{\sphinxupquote{richter\_correct}}}{\emph{\DUrole{n}{period}\DUrole{o}{=}\DUrole{default_value}{(None, None)}}}{}
\sphinxAtStartPar
Do the richter correction on the filled data for the given period.
\begin{quote}\begin{description}
\item[{Parameters}] \leavevmode
\sphinxAtStartPar
\sphinxstyleliteralstrong{\sphinxupquote{period}} ({\hyperref[\detokenize{weatherDB.lib:weatherDB.lib.utils.TimestampPeriod}]{\sphinxcrossref{\sphinxstyleliteralemphasis{\sphinxupquote{TimestampPeriod}}}}}\sphinxstyleliteralemphasis{\sphinxupquote{ or }}\sphinxstyleliteralemphasis{\sphinxupquote{(}}\sphinxstyleliteralemphasis{\sphinxupquote{tuple}}\sphinxstyleliteralemphasis{\sphinxupquote{ or }}\sphinxstyleliteralemphasis{\sphinxupquote{list of datetime.datetime}}\sphinxstyleliteralemphasis{\sphinxupquote{ or }}\sphinxstyleliteralemphasis{\sphinxupquote{None}}\sphinxstyleliteralemphasis{\sphinxupquote{)}}\sphinxstyleliteralemphasis{\sphinxupquote{, }}\sphinxstyleliteralemphasis{\sphinxupquote{optional}}) \textendash{} The minimum and maximum Timestamp for which to get the timeseries.
If None is given, the maximum or minimal possible Timestamp is taken.
The default is (None, None).

\item[{Raises}] \leavevmode
\sphinxAtStartPar
\sphinxstyleliteralstrong{\sphinxupquote{Exception}} \textendash{} If no richter class was found for this station.

\end{description}\end{quote}

\end{fulllineitems}

\index{update\_horizon() (weatherDB.station.PrecipitationStation method)@\spxentry{update\_horizon()}\spxextra{weatherDB.station.PrecipitationStation method}}

\begin{fulllineitems}
\phantomsection\label{\detokenize{weatherDB:weatherDB.station.PrecipitationStation.update_horizon}}\pysiglinewithargsret{\sphinxbfcode{\sphinxupquote{update\_horizon}}}{\emph{\DUrole{n}{skip\_if\_exist}\DUrole{o}{=}\DUrole{default_value}{True}}}{}
\sphinxAtStartPar
Update the horizon angle (Horizontabschirmung) in the meta table.

\sphinxAtStartPar
Get new values from the raster and put in the table.
\begin{quote}\begin{description}
\item[{Parameters}] \leavevmode
\sphinxAtStartPar
\sphinxstyleliteralstrong{\sphinxupquote{skip\_if\_exist}} (\sphinxstyleliteralemphasis{\sphinxupquote{bool}}\sphinxstyleliteralemphasis{\sphinxupquote{, }}\sphinxstyleliteralemphasis{\sphinxupquote{optional}}) \textendash{} Skip updating the value if there is already a value in the meta table.
The default is True.

\item[{Returns}] \leavevmode
\sphinxAtStartPar
The horizon angle in degrees (Horizontabschirmung).

\item[{Return type}] \leavevmode
\sphinxAtStartPar
float

\end{description}\end{quote}

\end{fulllineitems}

\index{update\_richter\_class() (weatherDB.station.PrecipitationStation method)@\spxentry{update\_richter\_class()}\spxextra{weatherDB.station.PrecipitationStation method}}

\begin{fulllineitems}
\phantomsection\label{\detokenize{weatherDB:weatherDB.station.PrecipitationStation.update_richter_class}}\pysiglinewithargsret{\sphinxbfcode{\sphinxupquote{update\_richter\_class}}}{\emph{\DUrole{n}{skip\_if\_exist}\DUrole{o}{=}\DUrole{default_value}{True}}}{}
\sphinxAtStartPar
Update the richter class in the meta table.

\sphinxAtStartPar
Get new values from the raster and put in the table.
\begin{quote}\begin{description}
\item[{Parameters}] \leavevmode
\sphinxAtStartPar
\sphinxstyleliteralstrong{\sphinxupquote{skip\_if\_exist}} (\sphinxstyleliteralemphasis{\sphinxupquote{bool}}\sphinxstyleliteralemphasis{\sphinxupquote{, }}\sphinxstyleliteralemphasis{\sphinxupquote{optional}}) \textendash{} Skip updating the value if there is already a value in the meta table.
The default is True

\item[{Returns}] \leavevmode
\sphinxAtStartPar
The richter class name.

\item[{Return type}] \leavevmode
\sphinxAtStartPar
str

\end{description}\end{quote}

\end{fulllineitems}


\end{fulllineitems}

\index{StationBase (class in weatherDB.station)@\spxentry{StationBase}\spxextra{class in weatherDB.station}}

\begin{fulllineitems}
\phantomsection\label{\detokenize{weatherDB:weatherDB.station.StationBase}}\pysiglinewithargsret{\sphinxbfcode{\sphinxupquote{class\DUrole{w}{  }}}\sphinxcode{\sphinxupquote{weatherDB.station.}}\sphinxbfcode{\sphinxupquote{StationBase}}}{\emph{\DUrole{n}{id}}}{}
\sphinxAtStartPar
Bases: \sphinxcode{\sphinxupquote{object}}

\sphinxAtStartPar
This is the Base class for one Station.
It is not working on it’s own, because those parameters need to get defined in the real classes
\index{\_\_init\_\_() (weatherDB.station.StationBase method)@\spxentry{\_\_init\_\_()}\spxextra{weatherDB.station.StationBase method}}

\begin{fulllineitems}
\phantomsection\label{\detokenize{weatherDB:weatherDB.station.StationBase.__init__}}\pysiglinewithargsret{\sphinxbfcode{\sphinxupquote{\_\_init\_\_}}}{\emph{\DUrole{n}{id}}}{}
\end{fulllineitems}

\index{download\_raw() (weatherDB.station.StationBase method)@\spxentry{download\_raw()}\spxextra{weatherDB.station.StationBase method}}

\begin{fulllineitems}
\phantomsection\label{\detokenize{weatherDB:weatherDB.station.StationBase.download_raw}}\pysiglinewithargsret{\sphinxbfcode{\sphinxupquote{download\_raw}}}{\emph{\DUrole{n}{only\_new}\DUrole{o}{=}\DUrole{default_value}{False}}}{}
\end{fulllineitems}

\index{fillup() (weatherDB.station.StationBase method)@\spxentry{fillup()}\spxextra{weatherDB.station.StationBase method}}

\begin{fulllineitems}
\phantomsection\label{\detokenize{weatherDB:weatherDB.station.StationBase.fillup}}\pysiglinewithargsret{\sphinxbfcode{\sphinxupquote{fillup}}}{\emph{\DUrole{n}{period}\DUrole{o}{=}\DUrole{default_value}{(None, None)}}}{}
\sphinxAtStartPar
Fill up missing data with measurements from nearby stations.

\end{fulllineitems}

\index{get\_adj() (weatherDB.station.StationBase method)@\spxentry{get\_adj()}\spxextra{weatherDB.station.StationBase method}}

\begin{fulllineitems}
\phantomsection\label{\detokenize{weatherDB:weatherDB.station.StationBase.get_adj}}\pysiglinewithargsret{\sphinxbfcode{\sphinxupquote{get\_adj}}}{\emph{\DUrole{n}{period}\DUrole{o}{=}\DUrole{default_value}{(None, None)}}, \emph{\DUrole{n}{agg\_to}\DUrole{o}{=}\DUrole{default_value}{None}}}{}
\sphinxAtStartPar
Get the adjusted timeserie.

\sphinxAtStartPar
The timeserie is adjusted to the multi annual mean.
So the overall mean of the given period will be the same as the multi annual mean.
\begin{quote}\begin{description}
\item[{Parameters}] \leavevmode\begin{itemize}
\item {} 
\sphinxAtStartPar
\sphinxstyleliteralstrong{\sphinxupquote{period}} ({\hyperref[\detokenize{weatherDB.lib:weatherDB.lib.utils.TimestampPeriod}]{\sphinxcrossref{\sphinxstyleliteralemphasis{\sphinxupquote{TimestampPeriod}}}}}\sphinxstyleliteralemphasis{\sphinxupquote{ or }}\sphinxstyleliteralemphasis{\sphinxupquote{(}}\sphinxstyleliteralemphasis{\sphinxupquote{tuple}}\sphinxstyleliteralemphasis{\sphinxupquote{ or }}\sphinxstyleliteralemphasis{\sphinxupquote{list of datetime.datetime}}\sphinxstyleliteralemphasis{\sphinxupquote{ or }}\sphinxstyleliteralemphasis{\sphinxupquote{None}}\sphinxstyleliteralemphasis{\sphinxupquote{)}}\sphinxstyleliteralemphasis{\sphinxupquote{, }}\sphinxstyleliteralemphasis{\sphinxupquote{optional}}) \textendash{} The minimum and maximum Timestamp for which to get the timeseries.
If None is given, the maximum or minimal possible Timestamp is taken.
The default is (None, None).

\item {} 
\sphinxAtStartPar
\sphinxstyleliteralstrong{\sphinxupquote{agg\_to}} (\sphinxstyleliteralemphasis{\sphinxupquote{str}}\sphinxstyleliteralemphasis{\sphinxupquote{ or }}\sphinxstyleliteralemphasis{\sphinxupquote{None}}\sphinxstyleliteralemphasis{\sphinxupquote{, }}\sphinxstyleliteralemphasis{\sphinxupquote{optional}}) \textendash{} Aggregate to a given timespan.
Can be anything smaller than the maximum timespan of the saved data.
If a Timeperiod smaller than the saved data is given, than the maximum possible timeperiod is returned.
For T and ET it can be “month”, “year”.
For N it can also be “hour”.
If None than the maximum timeperiod is taken.
The default is None.

\end{itemize}

\item[{Returns}] \leavevmode
\sphinxAtStartPar
A timeserie with the adjusted data.

\item[{Return type}] \leavevmode
\sphinxAtStartPar
pandas.DataFrame

\end{description}\end{quote}

\end{fulllineitems}

\index{get\_coef() (weatherDB.station.StationBase method)@\spxentry{get\_coef()}\spxextra{weatherDB.station.StationBase method}}

\begin{fulllineitems}
\phantomsection\label{\detokenize{weatherDB:weatherDB.station.StationBase.get_coef}}\pysiglinewithargsret{\sphinxbfcode{\sphinxupquote{get\_coef}}}{\emph{\DUrole{n}{other\_stid}}}{}
\sphinxAtStartPar
Get the regionalisation coefficients due to the height.

\sphinxAtStartPar
Those are the values from the dwd grid or regnie grids.
\begin{quote}\begin{description}
\item[{Parameters}] \leavevmode
\sphinxAtStartPar
\sphinxstyleliteralstrong{\sphinxupquote{other\_stid}} (\sphinxstyleliteralemphasis{\sphinxupquote{int}}) \textendash{} The Station Id of the other station from wich to regionalise for own station.

\item[{Returns}] \leavevmode
\sphinxAtStartPar
A list of

\item[{Return type}] \leavevmode
\sphinxAtStartPar
list of floats or None

\end{description}\end{quote}

\end{fulllineitems}

\index{get\_df() (weatherDB.station.StationBase method)@\spxentry{get\_df()}\spxextra{weatherDB.station.StationBase method}}

\begin{fulllineitems}
\phantomsection\label{\detokenize{weatherDB:weatherDB.station.StationBase.get_df}}\pysiglinewithargsret{\sphinxbfcode{\sphinxupquote{get\_df}}}{\emph{\DUrole{n}{kinds}}, \emph{\DUrole{n}{period}\DUrole{o}{=}\DUrole{default_value}{(None, None)}}, \emph{\DUrole{n}{agg\_to}\DUrole{o}{=}\DUrole{default_value}{None}}, \emph{\DUrole{n}{db\_unit}\DUrole{o}{=}\DUrole{default_value}{False}}}{}
\sphinxAtStartPar
Get a timeseries DataFrame from the database.
\begin{quote}\begin{description}
\item[{Parameters}] \leavevmode\begin{itemize}
\item {} 
\sphinxAtStartPar
\sphinxstyleliteralstrong{\sphinxupquote{kinds}} (\sphinxstyleliteralemphasis{\sphinxupquote{str}}\sphinxstyleliteralemphasis{\sphinxupquote{ or }}\sphinxstyleliteralemphasis{\sphinxupquote{list of str}}) \textendash{} The data kinds to update.
Must be a column in the timeseries DB.
Must be one of “raw”, “qc”, “filled”, “adj”.
For the precipitation also “qn” and “corr” are valid.

\item {} 
\sphinxAtStartPar
\sphinxstyleliteralstrong{\sphinxupquote{period}} ({\hyperref[\detokenize{weatherDB.lib:weatherDB.lib.utils.TimestampPeriod}]{\sphinxcrossref{\sphinxstyleliteralemphasis{\sphinxupquote{TimestampPeriod}}}}}\sphinxstyleliteralemphasis{\sphinxupquote{ or }}\sphinxstyleliteralemphasis{\sphinxupquote{(}}\sphinxstyleliteralemphasis{\sphinxupquote{tuple}}\sphinxstyleliteralemphasis{\sphinxupquote{ or }}\sphinxstyleliteralemphasis{\sphinxupquote{list of datetime.datetime}}\sphinxstyleliteralemphasis{\sphinxupquote{ or }}\sphinxstyleliteralemphasis{\sphinxupquote{None}}\sphinxstyleliteralemphasis{\sphinxupquote{)}}\sphinxstyleliteralemphasis{\sphinxupquote{, }}\sphinxstyleliteralemphasis{\sphinxupquote{optional}}) \textendash{} The minimum and maximum Timestamp for which to get the timeseries.
If None is given, the maximum or minimal possible Timestamp is taken.
The default is (None, None).

\item {} 
\sphinxAtStartPar
\sphinxstyleliteralstrong{\sphinxupquote{agg\_to}} (\sphinxstyleliteralemphasis{\sphinxupquote{str}}\sphinxstyleliteralemphasis{\sphinxupquote{ or }}\sphinxstyleliteralemphasis{\sphinxupquote{None}}\sphinxstyleliteralemphasis{\sphinxupquote{, }}\sphinxstyleliteralemphasis{\sphinxupquote{optional}}) \textendash{} Aggregate to a given timespan.
Can be anything smaller than the maximum timespan of the saved data.
If a Timeperiod smaller than the saved data is given, than the maximum possible timeperiod is returned.
For T and ET it can be “month”, “year”.
For N it can also be “hour”.
If None than the maximum timeperiod is taken.
The default is None.

\item {} 
\sphinxAtStartPar
\sphinxstyleliteralstrong{\sphinxupquote{db\_unit}} (\sphinxstyleliteralemphasis{\sphinxupquote{bool}}\sphinxstyleliteralemphasis{\sphinxupquote{, }}\sphinxstyleliteralemphasis{\sphinxupquote{optional}}) \textendash{} Should the result be in the Database unit.
If False the unit is getting converted to normal unit, like mm or °C.
The default is False.

\end{itemize}

\item[{Returns}] \leavevmode
\sphinxAtStartPar
The timeserie Dataframe with a DatetimeIndex.

\item[{Return type}] \leavevmode
\sphinxAtStartPar
pandas.DataFrame

\end{description}\end{quote}

\end{fulllineitems}

\index{get\_dist() (weatherDB.station.StationBase method)@\spxentry{get\_dist()}\spxextra{weatherDB.station.StationBase method}}

\begin{fulllineitems}
\phantomsection\label{\detokenize{weatherDB:weatherDB.station.StationBase.get_dist}}\pysiglinewithargsret{\sphinxbfcode{\sphinxupquote{get\_dist}}}{\emph{\DUrole{n}{period}\DUrole{o}{=}\DUrole{default_value}{(None, None)}}}{}
\sphinxAtStartPar
Get the timeserie with the infomation from which station the data got filled and the corresponding distance to this station.
\begin{quote}\begin{description}
\item[{Parameters}] \leavevmode
\sphinxAtStartPar
\sphinxstyleliteralstrong{\sphinxupquote{period}} ({\hyperref[\detokenize{weatherDB.lib:weatherDB.lib.utils.TimestampPeriod}]{\sphinxcrossref{\sphinxstyleliteralemphasis{\sphinxupquote{TimestampPeriod}}}}}\sphinxstyleliteralemphasis{\sphinxupquote{ or }}\sphinxstyleliteralemphasis{\sphinxupquote{(}}\sphinxstyleliteralemphasis{\sphinxupquote{tuple}}\sphinxstyleliteralemphasis{\sphinxupquote{ or }}\sphinxstyleliteralemphasis{\sphinxupquote{list of datetime.datetime}}\sphinxstyleliteralemphasis{\sphinxupquote{ or }}\sphinxstyleliteralemphasis{\sphinxupquote{None}}\sphinxstyleliteralemphasis{\sphinxupquote{)}}\sphinxstyleliteralemphasis{\sphinxupquote{, }}\sphinxstyleliteralemphasis{\sphinxupquote{optional}}) \textendash{} The minimum and maximum Timestamp for which to get the timeserie.
If None is given, the maximum or minimal possible Timestamp is taken.
The default is (None, None).

\item[{Returns}] \leavevmode
\sphinxAtStartPar
The timeserie for this station and the given period with the station\_id and the distance in meters from which the data got filled from.

\item[{Return type}] \leavevmode
\sphinxAtStartPar
pd.DataFrame

\end{description}\end{quote}

\end{fulllineitems}

\index{get\_filled() (weatherDB.station.StationBase method)@\spxentry{get\_filled()}\spxextra{weatherDB.station.StationBase method}}

\begin{fulllineitems}
\phantomsection\label{\detokenize{weatherDB:weatherDB.station.StationBase.get_filled}}\pysiglinewithargsret{\sphinxbfcode{\sphinxupquote{get\_filled}}}{\emph{\DUrole{n}{period}\DUrole{o}{=}\DUrole{default_value}{(None, None)}}, \emph{\DUrole{n}{with\_dist}\DUrole{o}{=}\DUrole{default_value}{False}}}{}
\sphinxAtStartPar
Get the filled timeserie.

\sphinxAtStartPar
Either only the timeserie is returned or also the id of the station from which the station data got filled, together with the distance to this station in m.
\begin{quote}\begin{description}
\item[{Parameters}] \leavevmode\begin{itemize}
\item {} 
\sphinxAtStartPar
\sphinxstyleliteralstrong{\sphinxupquote{period}} ({\hyperref[\detokenize{weatherDB.lib:weatherDB.lib.utils.TimestampPeriod}]{\sphinxcrossref{\sphinxstyleliteralemphasis{\sphinxupquote{TimestampPeriod}}}}}\sphinxstyleliteralemphasis{\sphinxupquote{ or }}\sphinxstyleliteralemphasis{\sphinxupquote{(}}\sphinxstyleliteralemphasis{\sphinxupquote{tuple}}\sphinxstyleliteralemphasis{\sphinxupquote{ or }}\sphinxstyleliteralemphasis{\sphinxupquote{list of datetime.datetime}}\sphinxstyleliteralemphasis{\sphinxupquote{ or }}\sphinxstyleliteralemphasis{\sphinxupquote{None}}\sphinxstyleliteralemphasis{\sphinxupquote{)}}\sphinxstyleliteralemphasis{\sphinxupquote{, }}\sphinxstyleliteralemphasis{\sphinxupquote{optional}}) \textendash{} The minimum and maximum Timestamp for which to get the timeserie.
If None is given, the maximum or minimal possible Timestamp is taken.
The default is (None, None).

\item {} 
\sphinxAtStartPar
\sphinxstyleliteralstrong{\sphinxupquote{with\_dist}} (\sphinxstyleliteralemphasis{\sphinxupquote{bool}}\sphinxstyleliteralemphasis{\sphinxupquote{, }}\sphinxstyleliteralemphasis{\sphinxupquote{optional}}) \textendash{} Should the distance to the stations from which the timeseries got filled be added.
The default is False.

\end{itemize}

\item[{Returns}] \leavevmode
\sphinxAtStartPar
The filled timeserie for this station and the given period.

\item[{Return type}] \leavevmode
\sphinxAtStartPar
pd.DataFrame

\end{description}\end{quote}

\end{fulllineitems}

\index{get\_filled\_period() (weatherDB.station.StationBase method)@\spxentry{get\_filled\_period()}\spxextra{weatherDB.station.StationBase method}}

\begin{fulllineitems}
\phantomsection\label{\detokenize{weatherDB:weatherDB.station.StationBase.get_filled_period}}\pysiglinewithargsret{\sphinxbfcode{\sphinxupquote{get\_filled\_period}}}{\emph{\DUrole{n}{kind}}, \emph{\DUrole{n}{from\_meta}\DUrole{o}{=}\DUrole{default_value}{False}}}{}
\sphinxAtStartPar
Get the min and max Timestamp for which there is data in the corresponding timeserie.

\sphinxAtStartPar
Computes the period from the timeserie or meta table.
\begin{quote}\begin{description}
\item[{Parameters}] \leavevmode\begin{itemize}
\item {} 
\sphinxAtStartPar
\sphinxstyleliteralstrong{\sphinxupquote{kind}} (\sphinxstyleliteralemphasis{\sphinxupquote{str}}) \textendash{} The data kind to look for filled period.
Must be a column in the timeseries DB.
Must be one of “raw”, “qc”, “filled”, “adj”.
If “best” is given, then depending on the parameter of the station the best kind is selected.
For Precipitation this is “corr” and for the other this is “filled”.
For the precipitation also “qn” and “corr” are valid.

\item {} 
\sphinxAtStartPar
\sphinxstyleliteralstrong{\sphinxupquote{from\_meta}} (\sphinxstyleliteralemphasis{\sphinxupquote{bool}}\sphinxstyleliteralemphasis{\sphinxupquote{, }}\sphinxstyleliteralemphasis{\sphinxupquote{optional}}) \textendash{} Should the period be from the meta table?
If False: the period is returned from the timeserie. In this case this function is only a wrapper for .get\_period\_meta.
The default is False.

\end{itemize}

\item[{Raises}] \leavevmode\begin{itemize}
\item {} 
\sphinxAtStartPar
\sphinxstyleliteralstrong{\sphinxupquote{NotImplementedError}} \textendash{} If the given kind is not valid.

\item {} 
\sphinxAtStartPar
\sphinxstyleliteralstrong{\sphinxupquote{ValueError}} \textendash{} If the given kind is not a string.

\end{itemize}

\item[{Returns}] \leavevmode
\sphinxAtStartPar
A TimestampPeriod of the filled timeserie.
(NaT, NaT) if the timeserie is all empty or not defined.

\item[{Return type}] \leavevmode
\sphinxAtStartPar
util.TimestampPeriod

\end{description}\end{quote}

\end{fulllineitems}

\index{get\_geom() (weatherDB.station.StationBase method)@\spxentry{get\_geom()}\spxextra{weatherDB.station.StationBase method}}

\begin{fulllineitems}
\phantomsection\label{\detokenize{weatherDB:weatherDB.station.StationBase.get_geom}}\pysiglinewithargsret{\sphinxbfcode{\sphinxupquote{get\_geom}}}{\emph{\DUrole{n}{format}\DUrole{o}{=}\DUrole{default_value}{\textquotesingle{}EWKT\textquotesingle{}}}, \emph{\DUrole{n}{crs}\DUrole{o}{=}\DUrole{default_value}{None}}}{}
\sphinxAtStartPar
Get the point geometry of the station.
\begin{quote}\begin{description}
\item[{Parameters}] \leavevmode\begin{itemize}
\item {} 
\sphinxAtStartPar
\sphinxstyleliteralstrong{\sphinxupquote{format}} (\sphinxstyleliteralemphasis{\sphinxupquote{str}}\sphinxstyleliteralemphasis{\sphinxupquote{ or }}\sphinxstyleliteralemphasis{\sphinxupquote{None}}\sphinxstyleliteralemphasis{\sphinxupquote{, }}\sphinxstyleliteralemphasis{\sphinxupquote{optional}}) \textendash{} The format of the geometry to return.
Needs to be a format that is understood by Postgresql.
ST\_AsXXXXX function needs to exist in postgresql language.
If None, then the binary representation is returned.
the default is “EWKT”.

\item {} 
\sphinxAtStartPar
\sphinxstyleliteralstrong{\sphinxupquote{crs}} (\sphinxstyleliteralemphasis{\sphinxupquote{str}}\sphinxstyleliteralemphasis{\sphinxupquote{, }}\sphinxstyleliteralemphasis{\sphinxupquote{int}}\sphinxstyleliteralemphasis{\sphinxupquote{ or }}\sphinxstyleliteralemphasis{\sphinxupquote{None}}\sphinxstyleliteralemphasis{\sphinxupquote{, }}\sphinxstyleliteralemphasis{\sphinxupquote{optional}}) \textendash{} If None, then the geometry is returned in WGS84 (EPSG:4326).
If string, then it should be one of “WGS84” or “UTM”.
If int, then it should be the EPSG code.

\end{itemize}

\item[{Returns}] \leavevmode
\sphinxAtStartPar
string or bytes representation of the geometry,
depending on the selected format.

\item[{Return type}] \leavevmode
\sphinxAtStartPar
str or bytes

\end{description}\end{quote}

\end{fulllineitems}

\index{get\_geom\_shp() (weatherDB.station.StationBase method)@\spxentry{get\_geom\_shp()}\spxextra{weatherDB.station.StationBase method}}

\begin{fulllineitems}
\phantomsection\label{\detokenize{weatherDB:weatherDB.station.StationBase.get_geom_shp}}\pysiglinewithargsret{\sphinxbfcode{\sphinxupquote{get\_geom\_shp}}}{\emph{\DUrole{n}{crs}\DUrole{o}{=}\DUrole{default_value}{None}}}{}
\sphinxAtStartPar
Get the geometry of the station as a shapely Point object.
\begin{quote}\begin{description}
\item[{Parameters}] \leavevmode
\sphinxAtStartPar
\sphinxstyleliteralstrong{\sphinxupquote{crs}} (\sphinxstyleliteralemphasis{\sphinxupquote{str}}\sphinxstyleliteralemphasis{\sphinxupquote{, }}\sphinxstyleliteralemphasis{\sphinxupquote{int}}\sphinxstyleliteralemphasis{\sphinxupquote{ or }}\sphinxstyleliteralemphasis{\sphinxupquote{None}}\sphinxstyleliteralemphasis{\sphinxupquote{, }}\sphinxstyleliteralemphasis{\sphinxupquote{optional}}) \textendash{} If None, then the geometry is returned in WGS84 (EPSG:4326).
If string, then it should be one of “WGS84” or “UTM”.
If int, then it should be the EPSG code.

\item[{Returns}] \leavevmode
\sphinxAtStartPar
The location of the station as shapely Point.

\item[{Return type}] \leavevmode
\sphinxAtStartPar
shapely.geometries.Point

\end{description}\end{quote}

\end{fulllineitems}

\index{get\_last\_imp\_period() (weatherDB.station.StationBase method)@\spxentry{get\_last\_imp\_period()}\spxextra{weatherDB.station.StationBase method}}

\begin{fulllineitems}
\phantomsection\label{\detokenize{weatherDB:weatherDB.station.StationBase.get_last_imp_period}}\pysiglinewithargsret{\sphinxbfcode{\sphinxupquote{get\_last\_imp\_period}}}{\emph{\DUrole{n}{all}\DUrole{o}{=}\DUrole{default_value}{False}}}{}
\sphinxAtStartPar
Get the last imported Period for this Station.
\begin{quote}\begin{description}
\item[{Parameters}] \leavevmode
\sphinxAtStartPar
\sphinxstyleliteralstrong{\sphinxupquote{all}} (\sphinxstyleliteralemphasis{\sphinxupquote{bool}}\sphinxstyleliteralemphasis{\sphinxupquote{, }}\sphinxstyleliteralemphasis{\sphinxupquote{optional}}) \textendash{} Should the maximum Timespan for all the last imports be returned.
If False only the period for this station is returned.
The default is False.

\item[{Returns}] \leavevmode
\sphinxAtStartPar
(minimal datetime, maximal datetime)

\item[{Return type}] \leavevmode
\sphinxAtStartPar
TimespanPeriod or tuple of datetime.datetime

\end{description}\end{quote}

\end{fulllineitems}

\index{get\_ma() (weatherDB.station.StationBase method)@\spxentry{get\_ma()}\spxextra{weatherDB.station.StationBase method}}

\begin{fulllineitems}
\phantomsection\label{\detokenize{weatherDB:weatherDB.station.StationBase.get_ma}}\pysiglinewithargsret{\sphinxbfcode{\sphinxupquote{get\_ma}}}{}{}
\end{fulllineitems}

\index{get\_meta() (weatherDB.station.StationBase method)@\spxentry{get\_meta()}\spxextra{weatherDB.station.StationBase method}}

\begin{fulllineitems}
\phantomsection\label{\detokenize{weatherDB:weatherDB.station.StationBase.get_meta}}\pysiglinewithargsret{\sphinxbfcode{\sphinxupquote{get\_meta}}}{\emph{\DUrole{n}{infos}\DUrole{o}{=}\DUrole{default_value}{\textquotesingle{}all\textquotesingle{}}}}{}
\sphinxAtStartPar
Get Informations from the meta table.
\begin{quote}\begin{description}
\item[{Parameters}] \leavevmode
\sphinxAtStartPar
\sphinxstyleliteralstrong{\sphinxupquote{infos}} (\sphinxstyleliteralemphasis{\sphinxupquote{list of str}}\sphinxstyleliteralemphasis{\sphinxupquote{ or }}\sphinxstyleliteralemphasis{\sphinxupquote{str}}\sphinxstyleliteralemphasis{\sphinxupquote{, }}\sphinxstyleliteralemphasis{\sphinxupquote{optional}}) \textendash{} A list of the informations to get from the database.
If “all” then all the informations are returned.
The default is “all”.

\item[{Returns}] \leavevmode
\sphinxAtStartPar
list with the informations.

\item[{Return type}] \leavevmode
\sphinxAtStartPar
list

\end{description}\end{quote}

\end{fulllineitems}

\index{get\_multi\_annual() (weatherDB.station.StationBase method)@\spxentry{get\_multi\_annual()}\spxextra{weatherDB.station.StationBase method}}

\begin{fulllineitems}
\phantomsection\label{\detokenize{weatherDB:weatherDB.station.StationBase.get_multi_annual}}\pysiglinewithargsret{\sphinxbfcode{\sphinxupquote{get\_multi\_annual}}}{}{}
\sphinxAtStartPar
Get the multi annual value(s) for this station.
\begin{quote}\begin{description}
\item[{Returns}] \leavevmode
\sphinxAtStartPar
The corresponding multi annual value.
For T en ET the yearly value is returned.
For N the winter and summer half yearly sum is returned in tuple.

\item[{Return type}] \leavevmode
\sphinxAtStartPar
list or number

\end{description}\end{quote}

\end{fulllineitems}

\index{get\_name() (weatherDB.station.StationBase method)@\spxentry{get\_name()}\spxextra{weatherDB.station.StationBase method}}

\begin{fulllineitems}
\phantomsection\label{\detokenize{weatherDB:weatherDB.station.StationBase.get_name}}\pysiglinewithargsret{\sphinxbfcode{\sphinxupquote{get\_name}}}{}{}
\end{fulllineitems}

\index{get\_neighboor\_stids() (weatherDB.station.StationBase method)@\spxentry{get\_neighboor\_stids()}\spxextra{weatherDB.station.StationBase method}}

\begin{fulllineitems}
\phantomsection\label{\detokenize{weatherDB:weatherDB.station.StationBase.get_neighboor_stids}}\pysiglinewithargsret{\sphinxbfcode{\sphinxupquote{get\_neighboor\_stids}}}{\emph{\DUrole{n}{n}\DUrole{o}{=}\DUrole{default_value}{5}}, \emph{\DUrole{n}{only\_real}\DUrole{o}{=}\DUrole{default_value}{True}}}{}
\sphinxAtStartPar
Get a list with Station Ids of the nearest neighboor stations.
\begin{quote}\begin{description}
\item[{Parameters}] \leavevmode\begin{itemize}
\item {} 
\sphinxAtStartPar
\sphinxstyleliteralstrong{\sphinxupquote{n}} (\sphinxstyleliteralemphasis{\sphinxupquote{int}}\sphinxstyleliteralemphasis{\sphinxupquote{, }}\sphinxstyleliteralemphasis{\sphinxupquote{optional}}) \textendash{} The number of stations to return.
If None, then all the possible stations are returned.
The default is 5.

\item {} 
\sphinxAtStartPar
\sphinxstyleliteralstrong{\sphinxupquote{only\_real}} (\sphinxstyleliteralemphasis{\sphinxupquote{bool}}\sphinxstyleliteralemphasis{\sphinxupquote{, }}\sphinxstyleliteralemphasis{\sphinxupquote{optional}}) \textendash{} Should only real station get considered?
If false also virtual stations are part of the result.
The default is True.

\end{itemize}

\item[{Returns}] \leavevmode
\sphinxAtStartPar
A list of station Ids in order of distance.
The closest station is the first in the list.

\item[{Return type}] \leavevmode
\sphinxAtStartPar
list of int

\end{description}\end{quote}

\end{fulllineitems}

\index{get\_period\_meta() (weatherDB.station.StationBase method)@\spxentry{get\_period\_meta()}\spxextra{weatherDB.station.StationBase method}}

\begin{fulllineitems}
\phantomsection\label{\detokenize{weatherDB:weatherDB.station.StationBase.get_period_meta}}\pysiglinewithargsret{\sphinxbfcode{\sphinxupquote{get\_period\_meta}}}{\emph{\DUrole{n}{kind}}, \emph{\DUrole{n}{all}\DUrole{o}{=}\DUrole{default_value}{False}}}{}
\sphinxAtStartPar
Get a specific period from the meta information table.

\sphinxAtStartPar
This functions returns the information from the meta table.
In this table there are several periods saved, like the period of the last import.
\begin{quote}\begin{description}
\item[{Parameters}] \leavevmode\begin{itemize}
\item {} 
\sphinxAtStartPar
\sphinxstyleliteralstrong{\sphinxupquote{kind}} (\sphinxstyleliteralemphasis{\sphinxupquote{str}}) \textendash{} The kind of period to return.
Should be one of {[}‘filled’, ‘raw’, ‘last\_imp’{]}.
filled: the maximum filled period of the filled timeserie.
raw: the maximum filled timeperiod of the raw data.
last\_imp: the maximum filled timeperiod of the last import.

\item {} 
\sphinxAtStartPar
\sphinxstyleliteralstrong{\sphinxupquote{all}} (\sphinxstyleliteralemphasis{\sphinxupquote{bool}}\sphinxstyleliteralemphasis{\sphinxupquote{, }}\sphinxstyleliteralemphasis{\sphinxupquote{optional}}) \textendash{} Should the maximum Timespan for all the filled periods be returned.
If False only the period for this station is returned.
The default is False.

\end{itemize}

\item[{Returns}] \leavevmode
\sphinxAtStartPar
The TimespanPeriod of the station or of all the stations if all=True.

\item[{Return type}] \leavevmode
\sphinxAtStartPar
TimespanPeriod

\item[{Raises}] \leavevmode
\sphinxAtStartPar
\sphinxstyleliteralstrong{\sphinxupquote{ValueError}} \textendash{} If a wrong kind is handed in.

\end{description}\end{quote}

\end{fulllineitems}

\index{get\_qc() (weatherDB.station.StationBase method)@\spxentry{get\_qc()}\spxextra{weatherDB.station.StationBase method}}

\begin{fulllineitems}
\phantomsection\label{\detokenize{weatherDB:weatherDB.station.StationBase.get_qc}}\pysiglinewithargsret{\sphinxbfcode{\sphinxupquote{get\_qc}}}{\emph{\DUrole{n}{period}\DUrole{o}{=}\DUrole{default_value}{(None, None)}}, \emph{\DUrole{n}{agg\_to}\DUrole{o}{=}\DUrole{default_value}{None}}}{}
\sphinxAtStartPar
Get the quality checked timeserie.
\begin{quote}\begin{description}
\item[{Parameters}] \leavevmode\begin{itemize}
\item {} 
\sphinxAtStartPar
\sphinxstyleliteralstrong{\sphinxupquote{period}} ({\hyperref[\detokenize{weatherDB.lib:weatherDB.lib.utils.TimestampPeriod}]{\sphinxcrossref{\sphinxstyleliteralemphasis{\sphinxupquote{TimestampPeriod}}}}}\sphinxstyleliteralemphasis{\sphinxupquote{ or }}\sphinxstyleliteralemphasis{\sphinxupquote{(}}\sphinxstyleliteralemphasis{\sphinxupquote{tuple}}\sphinxstyleliteralemphasis{\sphinxupquote{ or }}\sphinxstyleliteralemphasis{\sphinxupquote{list of datetime.datetime}}\sphinxstyleliteralemphasis{\sphinxupquote{ or }}\sphinxstyleliteralemphasis{\sphinxupquote{None}}\sphinxstyleliteralemphasis{\sphinxupquote{)}}\sphinxstyleliteralemphasis{\sphinxupquote{, }}\sphinxstyleliteralemphasis{\sphinxupquote{optional}}) \textendash{} The minimum and maximum Timestamp for which to get the timeserie.
If None is given, the maximum or minimal possible Timestamp is taken.
The default is (None, None).

\item {} 
\sphinxAtStartPar
\sphinxstyleliteralstrong{\sphinxupquote{agg\_to}} (\sphinxstyleliteralemphasis{\sphinxupquote{str}}\sphinxstyleliteralemphasis{\sphinxupquote{ or }}\sphinxstyleliteralemphasis{\sphinxupquote{None}}\sphinxstyleliteralemphasis{\sphinxupquote{, }}\sphinxstyleliteralemphasis{\sphinxupquote{optional}}) \textendash{} Aggregate to a given timespan.
Can be anything smaller than the maximum timespan of the saved data.
If a Timeperiod smaller than the saved data is given, than the maximum possible timeperiod is returned.
For T and ET it can be “month”, “year”.
For N it can also be “hour”.
If None than the maximum timeperiod is taken.
The default is None.

\end{itemize}

\item[{Returns}] \leavevmode
\sphinxAtStartPar
The quality checked timeserie for this station and the given period.

\item[{Return type}] \leavevmode
\sphinxAtStartPar
pd.DataFrame

\end{description}\end{quote}

\end{fulllineitems}

\index{get\_raster\_value() (weatherDB.station.StationBase method)@\spxentry{get\_raster\_value()}\spxextra{weatherDB.station.StationBase method}}

\begin{fulllineitems}
\phantomsection\label{\detokenize{weatherDB:weatherDB.station.StationBase.get_raster_value}}\pysiglinewithargsret{\sphinxbfcode{\sphinxupquote{get\_raster\_value}}}{\emph{\DUrole{n}{raster}}}{}
\end{fulllineitems}

\index{get\_raw() (weatherDB.station.StationBase method)@\spxentry{get\_raw()}\spxextra{weatherDB.station.StationBase method}}

\begin{fulllineitems}
\phantomsection\label{\detokenize{weatherDB:weatherDB.station.StationBase.get_raw}}\pysiglinewithargsret{\sphinxbfcode{\sphinxupquote{get\_raw}}}{\emph{\DUrole{n}{period}\DUrole{o}{=}\DUrole{default_value}{(None, None)}}, \emph{\DUrole{n}{agg\_to}\DUrole{o}{=}\DUrole{default_value}{None}}}{}
\sphinxAtStartPar
Get the raw timeserie.
\begin{quote}\begin{description}
\item[{Parameters}] \leavevmode\begin{itemize}
\item {} 
\sphinxAtStartPar
\sphinxstyleliteralstrong{\sphinxupquote{period}} ({\hyperref[\detokenize{weatherDB.lib:weatherDB.lib.utils.TimestampPeriod}]{\sphinxcrossref{\sphinxstyleliteralemphasis{\sphinxupquote{TimestampPeriod}}}}}\sphinxstyleliteralemphasis{\sphinxupquote{ or }}\sphinxstyleliteralemphasis{\sphinxupquote{(}}\sphinxstyleliteralemphasis{\sphinxupquote{tuple}}\sphinxstyleliteralemphasis{\sphinxupquote{ or }}\sphinxstyleliteralemphasis{\sphinxupquote{list of datetime.datetime}}\sphinxstyleliteralemphasis{\sphinxupquote{ or }}\sphinxstyleliteralemphasis{\sphinxupquote{None}}\sphinxstyleliteralemphasis{\sphinxupquote{)}}\sphinxstyleliteralemphasis{\sphinxupquote{, }}\sphinxstyleliteralemphasis{\sphinxupquote{optional}}) \textendash{} The minimum and maximum Timestamp for which to get the timeserie.
If None is given, the maximum or minimal possible Timestamp is taken.
The default is (None, None).

\item {} 
\sphinxAtStartPar
\sphinxstyleliteralstrong{\sphinxupquote{agg\_to}} (\sphinxstyleliteralemphasis{\sphinxupquote{str}}\sphinxstyleliteralemphasis{\sphinxupquote{ or }}\sphinxstyleliteralemphasis{\sphinxupquote{None}}\sphinxstyleliteralemphasis{\sphinxupquote{, }}\sphinxstyleliteralemphasis{\sphinxupquote{optional}}) \textendash{} Aggregate to a given timespan.
Can be anything smaller than the maximum timespan of the saved data.
If a Timeperiod smaller than the saved data is given, than the maximum possible timeperiod is returned.
For T and ET it can be “month”, “year”.
For N it can also be “hour”.
If None than the maximum timeperiod is taken.
The default is None.

\end{itemize}

\item[{Returns}] \leavevmode
\sphinxAtStartPar
The raw timeserie for this station and the given period.

\item[{Return type}] \leavevmode
\sphinxAtStartPar
pd.DataFrame

\end{description}\end{quote}

\end{fulllineitems}

\index{get\_zipfiles() (weatherDB.station.StationBase method)@\spxentry{get\_zipfiles()}\spxextra{weatherDB.station.StationBase method}}

\begin{fulllineitems}
\phantomsection\label{\detokenize{weatherDB:weatherDB.station.StationBase.get_zipfiles}}\pysiglinewithargsret{\sphinxbfcode{\sphinxupquote{get\_zipfiles}}}{\emph{\DUrole{n}{only\_new}\DUrole{o}{=}\DUrole{default_value}{True}}, \emph{\DUrole{n}{ftp\_file\_list}\DUrole{o}{=}\DUrole{default_value}{None}}}{}
\sphinxAtStartPar
Get the zipfiles on the CDC server with the raw data.
\begin{quote}\begin{description}
\item[{Parameters}] \leavevmode\begin{itemize}
\item {} 
\sphinxAtStartPar
\sphinxstyleliteralstrong{\sphinxupquote{only\_new}} (\sphinxstyleliteralemphasis{\sphinxupquote{bool}}\sphinxstyleliteralemphasis{\sphinxupquote{, }}\sphinxstyleliteralemphasis{\sphinxupquote{optional}}) \textendash{} Get only the files that are not yet in the database?
If False all the available files are loaded again.
The default is True

\item {} 
\sphinxAtStartPar
\sphinxstyleliteralstrong{\sphinxupquote{ftp\_file\_list}} (\sphinxstyleliteralemphasis{\sphinxupquote{list of}}\sphinxstyleliteralemphasis{\sphinxupquote{ (}}\sphinxstyleliteralemphasis{\sphinxupquote{strings}}\sphinxstyleliteralemphasis{\sphinxupquote{, }}\sphinxstyleliteralemphasis{\sphinxupquote{datetime}}\sphinxstyleliteralemphasis{\sphinxupquote{)}}\sphinxstyleliteralemphasis{\sphinxupquote{, }}\sphinxstyleliteralemphasis{\sphinxupquote{optional}}) \textendash{} A list of files on the FTP server together with their modification time.
If None, then the list is fetched from the server.
The default is None

\end{itemize}

\item[{Returns}] \leavevmode
\sphinxAtStartPar
A DataFrame of zipfiles and the corresponding modification time on the CDC server to import.

\item[{Return type}] \leavevmode
\sphinxAtStartPar
pandas.DataFrame or None

\end{description}\end{quote}

\end{fulllineitems}

\index{is\_last\_imp\_done() (weatherDB.station.StationBase method)@\spxentry{is\_last\_imp\_done()}\spxextra{weatherDB.station.StationBase method}}

\begin{fulllineitems}
\phantomsection\label{\detokenize{weatherDB:weatherDB.station.StationBase.is_last_imp_done}}\pysiglinewithargsret{\sphinxbfcode{\sphinxupquote{is\_last\_imp\_done}}}{\emph{\DUrole{n}{kind}}}{}
\sphinxAtStartPar
Is the last import for the given kind already worked in?
\begin{quote}\begin{description}
\item[{Parameters}] \leavevmode
\sphinxAtStartPar
\sphinxstyleliteralstrong{\sphinxupquote{kind}} (\sphinxstyleliteralemphasis{\sphinxupquote{str}}) \textendash{} The data kind to look for filled period.
Must be a column in the timeseries DB.
Must be one of “raw”, “qc”, “filled”, “adj”, “best”.
If “best” is given, then depending on the parameter of the station the best kind is selected.
For Precipitation this is “corr” and for the other this is “filled”.
For the precipitation also “qn” and “corr” are valid.

\item[{Returns}] \leavevmode
\sphinxAtStartPar
True if the last import of the given kind is already treated.

\item[{Return type}] \leavevmode
\sphinxAtStartPar
bool

\end{description}\end{quote}

\end{fulllineitems}

\index{is\_virtual() (weatherDB.station.StationBase method)@\spxentry{is\_virtual()}\spxextra{weatherDB.station.StationBase method}}

\begin{fulllineitems}
\phantomsection\label{\detokenize{weatherDB:weatherDB.station.StationBase.is_virtual}}\pysiglinewithargsret{\sphinxbfcode{\sphinxupquote{is\_virtual}}}{}{}
\sphinxAtStartPar
Check if the station is a real station or only a virtual one.

\sphinxAtStartPar
Real means that the DWD is measuring here.
Virtual means, that there are no measurements here, but the station got created to have timeseries for every parameter for every precipitation station.
\begin{quote}\begin{description}
\item[{Returns}] \leavevmode
\sphinxAtStartPar
true if the station is virtual, false if it is real.

\item[{Return type}] \leavevmode
\sphinxAtStartPar
bool

\end{description}\end{quote}

\end{fulllineitems}

\index{isin\_db() (weatherDB.station.StationBase method)@\spxentry{isin\_db()}\spxextra{weatherDB.station.StationBase method}}

\begin{fulllineitems}
\phantomsection\label{\detokenize{weatherDB:weatherDB.station.StationBase.isin_db}}\pysiglinewithargsret{\sphinxbfcode{\sphinxupquote{isin\_db}}}{}{}
\sphinxAtStartPar
Check if Station is already in a timeseries table.
\begin{quote}\begin{description}
\item[{Returns}] \leavevmode
\sphinxAtStartPar
True if Station has a table in DB, no matter if it is filled or not.

\item[{Return type}] \leavevmode
\sphinxAtStartPar
bool

\end{description}\end{quote}

\end{fulllineitems}

\index{isin\_ma() (weatherDB.station.StationBase method)@\spxentry{isin\_ma()}\spxextra{weatherDB.station.StationBase method}}

\begin{fulllineitems}
\phantomsection\label{\detokenize{weatherDB:weatherDB.station.StationBase.isin_ma}}\pysiglinewithargsret{\sphinxbfcode{\sphinxupquote{isin\_ma}}}{}{}
\sphinxAtStartPar
Check if Station is already in the multi annual table.
\begin{quote}\begin{description}
\item[{Returns}] \leavevmode
\sphinxAtStartPar
True if Station is in multi annual table.

\item[{Return type}] \leavevmode
\sphinxAtStartPar
bool

\end{description}\end{quote}

\end{fulllineitems}

\index{isin\_meta() (weatherDB.station.StationBase method)@\spxentry{isin\_meta()}\spxextra{weatherDB.station.StationBase method}}

\begin{fulllineitems}
\phantomsection\label{\detokenize{weatherDB:weatherDB.station.StationBase.isin_meta}}\pysiglinewithargsret{\sphinxbfcode{\sphinxupquote{isin\_meta}}}{}{}
\sphinxAtStartPar
Check if Station is already in the meta table.
\begin{quote}\begin{description}
\item[{Returns}] \leavevmode
\sphinxAtStartPar
True if Station is in meta table.

\item[{Return type}] \leavevmode
\sphinxAtStartPar
bool

\end{description}\end{quote}

\end{fulllineitems}

\index{last\_imp\_fillup() (weatherDB.station.StationBase method)@\spxentry{last\_imp\_fillup()}\spxextra{weatherDB.station.StationBase method}}

\begin{fulllineitems}
\phantomsection\label{\detokenize{weatherDB:weatherDB.station.StationBase.last_imp_fillup}}\pysiglinewithargsret{\sphinxbfcode{\sphinxupquote{last\_imp\_fillup}}}{\emph{\DUrole{n}{last\_imp\_period}\DUrole{o}{=}\DUrole{default_value}{None}}}{}
\sphinxAtStartPar
Do the filling up of the last import.

\end{fulllineitems}

\index{last\_imp\_qc() (weatherDB.station.StationBase method)@\spxentry{last\_imp\_qc()}\spxextra{weatherDB.station.StationBase method}}

\begin{fulllineitems}
\phantomsection\label{\detokenize{weatherDB:weatherDB.station.StationBase.last_imp_qc}}\pysiglinewithargsret{\sphinxbfcode{\sphinxupquote{last\_imp\_qc}}}{}{}
\end{fulllineitems}

\index{last\_imp\_quality\_check() (weatherDB.station.StationBase method)@\spxentry{last\_imp\_quality\_check()}\spxextra{weatherDB.station.StationBase method}}

\begin{fulllineitems}
\phantomsection\label{\detokenize{weatherDB:weatherDB.station.StationBase.last_imp_quality_check}}\pysiglinewithargsret{\sphinxbfcode{\sphinxupquote{last\_imp\_quality\_check}}}{}{}
\sphinxAtStartPar
Do the quality check of the last import.

\end{fulllineitems}

\index{plot() (weatherDB.station.StationBase method)@\spxentry{plot()}\spxextra{weatherDB.station.StationBase method}}

\begin{fulllineitems}
\phantomsection\label{\detokenize{weatherDB:weatherDB.station.StationBase.plot}}\pysiglinewithargsret{\sphinxbfcode{\sphinxupquote{plot}}}{\emph{\DUrole{n}{period}\DUrole{o}{=}\DUrole{default_value}{(None, None)}}, \emph{\DUrole{n}{kind}\DUrole{o}{=}\DUrole{default_value}{\textquotesingle{}filled\textquotesingle{}}}, \emph{\DUrole{n}{agg\_to}\DUrole{o}{=}\DUrole{default_value}{None}}, \emph{\DUrole{o}{**}\DUrole{n}{kwargs}}}{}
\sphinxAtStartPar
Plot the data of this station.
\begin{quote}\begin{description}
\item[{Parameters}] \leavevmode\begin{itemize}
\item {} 
\sphinxAtStartPar
\sphinxstyleliteralstrong{\sphinxupquote{period}} ({\hyperref[\detokenize{weatherDB.lib:weatherDB.lib.utils.TimestampPeriod}]{\sphinxcrossref{\sphinxstyleliteralemphasis{\sphinxupquote{TimestampPeriod}}}}}\sphinxstyleliteralemphasis{\sphinxupquote{ or }}\sphinxstyleliteralemphasis{\sphinxupquote{(}}\sphinxstyleliteralemphasis{\sphinxupquote{tuple}}\sphinxstyleliteralemphasis{\sphinxupquote{ or }}\sphinxstyleliteralemphasis{\sphinxupquote{list of datetime.datetime}}\sphinxstyleliteralemphasis{\sphinxupquote{ or }}\sphinxstyleliteralemphasis{\sphinxupquote{None}}\sphinxstyleliteralemphasis{\sphinxupquote{)}}\sphinxstyleliteralemphasis{\sphinxupquote{, }}\sphinxstyleliteralemphasis{\sphinxupquote{optional}}) \textendash{} The minimum and maximum Timestamp for which to get the timeseries.
If None is given, the maximum or minimal possible Timestamp is taken.
The default is (None, None).

\item {} 
\sphinxAtStartPar
\sphinxstyleliteralstrong{\sphinxupquote{kind}} (\sphinxstyleliteralemphasis{\sphinxupquote{str}}\sphinxstyleliteralemphasis{\sphinxupquote{, }}\sphinxstyleliteralemphasis{\sphinxupquote{optional}}) \textendash{} The data kind to plot.
Must be a column in the timeseries DB.
Must be one of “raw”, “qc”, “filled”, “adj”.
For the precipitation also “qn” and “corr” are valid.
The default is “filled.

\item {} 
\sphinxAtStartPar
\sphinxstyleliteralstrong{\sphinxupquote{agg\_to}} (\sphinxstyleliteralemphasis{\sphinxupquote{str}}\sphinxstyleliteralemphasis{\sphinxupquote{ or }}\sphinxstyleliteralemphasis{\sphinxupquote{None}}\sphinxstyleliteralemphasis{\sphinxupquote{, }}\sphinxstyleliteralemphasis{\sphinxupquote{optional}}) \textendash{} Aggregate to a given timespan.
Can be anything smaller than the maximum timespan of the saved data.
If a Timeperiod smaller than the saved data is given, than the maximum possible timeperiod is returned.
For T and ET it can be “month”, “year”.
For N it can also be “hour”.
If None than the maximum timeperiod is taken.
The default is None.

\end{itemize}

\end{description}\end{quote}

\end{fulllineitems}

\index{quality\_check() (weatherDB.station.StationBase method)@\spxentry{quality\_check()}\spxextra{weatherDB.station.StationBase method}}

\begin{fulllineitems}
\phantomsection\label{\detokenize{weatherDB:weatherDB.station.StationBase.quality_check}}\pysiglinewithargsret{\sphinxbfcode{\sphinxupquote{quality\_check}}}{\emph{\DUrole{n}{period}\DUrole{o}{=}\DUrole{default_value}{(None, None)}}}{}
\sphinxAtStartPar
Quality check the raw data for a given period.
\begin{quote}\begin{description}
\item[{Parameters}] \leavevmode
\sphinxAtStartPar
\sphinxstyleliteralstrong{\sphinxupquote{period}} (\sphinxstyleliteralemphasis{\sphinxupquote{util.TimestampPeriod}}\sphinxstyleliteralemphasis{\sphinxupquote{ or }}\sphinxstyleliteralemphasis{\sphinxupquote{(}}\sphinxstyleliteralemphasis{\sphinxupquote{tuple}}\sphinxstyleliteralemphasis{\sphinxupquote{ or }}\sphinxstyleliteralemphasis{\sphinxupquote{list of datetime.datetime}}\sphinxstyleliteralemphasis{\sphinxupquote{ or }}\sphinxstyleliteralemphasis{\sphinxupquote{None}}\sphinxstyleliteralemphasis{\sphinxupquote{)}}\sphinxstyleliteralemphasis{\sphinxupquote{, }}\sphinxstyleliteralemphasis{\sphinxupquote{optional}}) \textendash{} The minimum and maximum Timestamp for which to get the timeseries.
If None is given, the maximum or minimal possible Timestamp is taken.
The default is (None, None).

\end{description}\end{quote}

\end{fulllineitems}

\index{update\_ma() (weatherDB.station.StationBase method)@\spxentry{update\_ma()}\spxextra{weatherDB.station.StationBase method}}

\begin{fulllineitems}
\phantomsection\label{\detokenize{weatherDB:weatherDB.station.StationBase.update_ma}}\pysiglinewithargsret{\sphinxbfcode{\sphinxupquote{update\_ma}}}{\emph{\DUrole{n}{skip\_if\_exist}\DUrole{o}{=}\DUrole{default_value}{True}}}{}
\sphinxAtStartPar
Update the multi annual values in the stations\_raster\_values table.

\sphinxAtStartPar
Get new values from the raster and put in the table.

\end{fulllineitems}

\index{update\_period\_meta() (weatherDB.station.StationBase method)@\spxentry{update\_period\_meta()}\spxextra{weatherDB.station.StationBase method}}

\begin{fulllineitems}
\phantomsection\label{\detokenize{weatherDB:weatherDB.station.StationBase.update_period_meta}}\pysiglinewithargsret{\sphinxbfcode{\sphinxupquote{update\_period\_meta}}}{\emph{\DUrole{n}{kind}}}{}
\sphinxAtStartPar
Update the time period in the meta file.

\sphinxAtStartPar
Compute teh filled period of a timeserie and save in the meta table.
\begin{quote}\begin{description}
\item[{Parameters}] \leavevmode
\sphinxAtStartPar
\sphinxstyleliteralstrong{\sphinxupquote{kind}} (\sphinxstyleliteralemphasis{\sphinxupquote{str}}) \textendash{} The data kind to look for filled period.
Must be a column in the timeseries DB.
Must be one of “raw”, “qc”, “filled”.
If “best” is given, then depending on the parameter of the station the best kind is selected.
For Precipitation this is “corr” and for the other this is “filled”.
For the precipitation also “corr” are valid.

\end{description}\end{quote}

\end{fulllineitems}

\index{update\_raw() (weatherDB.station.StationBase method)@\spxentry{update\_raw()}\spxextra{weatherDB.station.StationBase method}}

\begin{fulllineitems}
\phantomsection\label{\detokenize{weatherDB:weatherDB.station.StationBase.update_raw}}\pysiglinewithargsret{\sphinxbfcode{\sphinxupquote{update\_raw}}}{\emph{\DUrole{n}{only\_new}\DUrole{o}{=}\DUrole{default_value}{True}}, \emph{\DUrole{n}{ftp\_file\_list}\DUrole{o}{=}\DUrole{default_value}{None}}}{}
\sphinxAtStartPar
Download data from CDC and upload to database.
\begin{quote}\begin{description}
\item[{Parameters}] \leavevmode\begin{itemize}
\item {} 
\sphinxAtStartPar
\sphinxstyleliteralstrong{\sphinxupquote{only\_new}} (\sphinxstyleliteralemphasis{\sphinxupquote{bool}}\sphinxstyleliteralemphasis{\sphinxupquote{, }}\sphinxstyleliteralemphasis{\sphinxupquote{optional}}) \textendash{} Get only the files that are not yet in the database?
If False all the available files are loaded again.
The default is True

\item {} 
\sphinxAtStartPar
\sphinxstyleliteralstrong{\sphinxupquote{ftp\_file\_list}} (\sphinxstyleliteralemphasis{\sphinxupquote{list of}}\sphinxstyleliteralemphasis{\sphinxupquote{ (}}\sphinxstyleliteralemphasis{\sphinxupquote{strings}}\sphinxstyleliteralemphasis{\sphinxupquote{, }}\sphinxstyleliteralemphasis{\sphinxupquote{datetime}}\sphinxstyleliteralemphasis{\sphinxupquote{)}}\sphinxstyleliteralemphasis{\sphinxupquote{, }}\sphinxstyleliteralemphasis{\sphinxupquote{optional}}) \textendash{} A list of files on the FTP server together with their modification time.
If None, then the list is fetched from the server.
The default is None

\end{itemize}

\item[{Returns}] \leavevmode
\sphinxAtStartPar
The raw Dataframe of the Stations data.

\item[{Return type}] \leavevmode
\sphinxAtStartPar
pandas.DataFrame

\end{description}\end{quote}

\end{fulllineitems}


\end{fulllineitems}

\index{StationET (in module weatherDB.station)@\spxentry{StationET}\spxextra{in module weatherDB.station}}

\begin{fulllineitems}
\phantomsection\label{\detokenize{weatherDB:weatherDB.station.StationET}}\pysigline{\sphinxcode{\sphinxupquote{weatherDB.station.}}\sphinxbfcode{\sphinxupquote{StationET}}}
\sphinxAtStartPar
alias of {\hyperref[\detokenize{weatherDB:weatherDB.station.EvapotranspirationStation}]{\sphinxcrossref{\sphinxcode{\sphinxupquote{weatherDB.station.EvapotranspirationStation}}}}}

\end{fulllineitems}

\index{StationN (in module weatherDB.station)@\spxentry{StationN}\spxextra{in module weatherDB.station}}

\begin{fulllineitems}
\phantomsection\label{\detokenize{weatherDB:weatherDB.station.StationN}}\pysigline{\sphinxcode{\sphinxupquote{weatherDB.station.}}\sphinxbfcode{\sphinxupquote{StationN}}}
\sphinxAtStartPar
alias of {\hyperref[\detokenize{weatherDB:weatherDB.station.PrecipitationStation}]{\sphinxcrossref{\sphinxcode{\sphinxupquote{weatherDB.station.PrecipitationStation}}}}}

\end{fulllineitems}

\index{StationNBase (class in weatherDB.station)@\spxentry{StationNBase}\spxextra{class in weatherDB.station}}

\begin{fulllineitems}
\phantomsection\label{\detokenize{weatherDB:weatherDB.station.StationNBase}}\pysiglinewithargsret{\sphinxbfcode{\sphinxupquote{class\DUrole{w}{  }}}\sphinxcode{\sphinxupquote{weatherDB.station.}}\sphinxbfcode{\sphinxupquote{StationNBase}}}{\emph{\DUrole{n}{id}}}{}
\sphinxAtStartPar
Bases: {\hyperref[\detokenize{weatherDB:weatherDB.station.StationBase}]{\sphinxcrossref{\sphinxcode{\sphinxupquote{weatherDB.station.StationBase}}}}}
\index{get\_adj() (weatherDB.station.StationNBase method)@\spxentry{get\_adj()}\spxextra{weatherDB.station.StationNBase method}}

\begin{fulllineitems}
\phantomsection\label{\detokenize{weatherDB:weatherDB.station.StationNBase.get_adj}}\pysiglinewithargsret{\sphinxbfcode{\sphinxupquote{get\_adj}}}{\emph{\DUrole{n}{period}\DUrole{o}{=}\DUrole{default_value}{(None, None)}}}{}
\sphinxAtStartPar
Get the adjusted timeserie.

\sphinxAtStartPar
The timeserie is adjusted to the multi annual mean.
So the overall mean of the given period will be the same as the multi annual mean.
\begin{quote}\begin{description}
\item[{Parameters}] \leavevmode\begin{itemize}
\item {} 
\sphinxAtStartPar
\sphinxstyleliteralstrong{\sphinxupquote{period}} ({\hyperref[\detokenize{weatherDB.lib:weatherDB.lib.utils.TimestampPeriod}]{\sphinxcrossref{\sphinxstyleliteralemphasis{\sphinxupquote{TimestampPeriod}}}}}\sphinxstyleliteralemphasis{\sphinxupquote{ or }}\sphinxstyleliteralemphasis{\sphinxupquote{(}}\sphinxstyleliteralemphasis{\sphinxupquote{tuple}}\sphinxstyleliteralemphasis{\sphinxupquote{ or }}\sphinxstyleliteralemphasis{\sphinxupquote{list of datetime.datetime}}\sphinxstyleliteralemphasis{\sphinxupquote{ or }}\sphinxstyleliteralemphasis{\sphinxupquote{None}}\sphinxstyleliteralemphasis{\sphinxupquote{)}}\sphinxstyleliteralemphasis{\sphinxupquote{, }}\sphinxstyleliteralemphasis{\sphinxupquote{optional}}) \textendash{} The minimum and maximum Timestamp for which to get the timeseries.
If None is given, the maximum or minimal possible Timestamp is taken.
The default is (None, None).

\item {} 
\sphinxAtStartPar
\sphinxstyleliteralstrong{\sphinxupquote{agg\_to}} (\sphinxstyleliteralemphasis{\sphinxupquote{str}}\sphinxstyleliteralemphasis{\sphinxupquote{ or }}\sphinxstyleliteralemphasis{\sphinxupquote{None}}\sphinxstyleliteralemphasis{\sphinxupquote{, }}\sphinxstyleliteralemphasis{\sphinxupquote{optional}}) \textendash{} Aggregate to a given timespan.
Can be anything smaller than the maximum timespan of the saved data.
If a Timeperiod smaller than the saved data is given, than the maximum possible timeperiod is returned.
For T and ET it can be “month”, “year”.
For N it can also be “hour”.
If None than the maximum timeperiod is taken.
The default is None.

\end{itemize}

\item[{Returns}] \leavevmode
\sphinxAtStartPar
A timeserie with the adjusted data.

\item[{Return type}] \leavevmode
\sphinxAtStartPar
pandas.DataFrame

\end{description}\end{quote}

\end{fulllineitems}


\end{fulllineitems}

\index{StationND (in module weatherDB.station)@\spxentry{StationND}\spxextra{in module weatherDB.station}}

\begin{fulllineitems}
\phantomsection\label{\detokenize{weatherDB:weatherDB.station.StationND}}\pysigline{\sphinxcode{\sphinxupquote{weatherDB.station.}}\sphinxbfcode{\sphinxupquote{StationND}}}
\sphinxAtStartPar
alias of {\hyperref[\detokenize{weatherDB:weatherDB.station.PrecipitationDailyStation}]{\sphinxcrossref{\sphinxcode{\sphinxupquote{weatherDB.station.PrecipitationDailyStation}}}}}

\end{fulllineitems}

\index{StationT (in module weatherDB.station)@\spxentry{StationT}\spxextra{in module weatherDB.station}}

\begin{fulllineitems}
\phantomsection\label{\detokenize{weatherDB:weatherDB.station.StationT}}\pysigline{\sphinxcode{\sphinxupquote{weatherDB.station.}}\sphinxbfcode{\sphinxupquote{StationT}}}
\sphinxAtStartPar
alias of {\hyperref[\detokenize{weatherDB:weatherDB.station.TemperatureStation}]{\sphinxcrossref{\sphinxcode{\sphinxupquote{weatherDB.station.TemperatureStation}}}}}

\end{fulllineitems}

\index{StationTETBase (class in weatherDB.station)@\spxentry{StationTETBase}\spxextra{class in weatherDB.station}}

\begin{fulllineitems}
\phantomsection\label{\detokenize{weatherDB:weatherDB.station.StationTETBase}}\pysiglinewithargsret{\sphinxbfcode{\sphinxupquote{class\DUrole{w}{  }}}\sphinxcode{\sphinxupquote{weatherDB.station.}}\sphinxbfcode{\sphinxupquote{StationTETBase}}}{\emph{\DUrole{n}{id}}}{}
\sphinxAtStartPar
Bases: {\hyperref[\detokenize{weatherDB:weatherDB.station.StationBase}]{\sphinxcrossref{\sphinxcode{\sphinxupquote{weatherDB.station.StationBase}}}}}
\index{get\_adj() (weatherDB.station.StationTETBase method)@\spxentry{get\_adj()}\spxextra{weatherDB.station.StationTETBase method}}

\begin{fulllineitems}
\phantomsection\label{\detokenize{weatherDB:weatherDB.station.StationTETBase.get_adj}}\pysiglinewithargsret{\sphinxbfcode{\sphinxupquote{get\_adj}}}{\emph{\DUrole{n}{period}\DUrole{o}{=}\DUrole{default_value}{(None, None)}}}{}
\sphinxAtStartPar
Get the adjusted timeserie.

\sphinxAtStartPar
The timeserie is adjusted to the multi annual mean.
So the overall mean of the given period will be the same as the multi annual mean.
\begin{quote}\begin{description}
\item[{Parameters}] \leavevmode\begin{itemize}
\item {} 
\sphinxAtStartPar
\sphinxstyleliteralstrong{\sphinxupquote{period}} ({\hyperref[\detokenize{weatherDB.lib:weatherDB.lib.utils.TimestampPeriod}]{\sphinxcrossref{\sphinxstyleliteralemphasis{\sphinxupquote{TimestampPeriod}}}}}\sphinxstyleliteralemphasis{\sphinxupquote{ or }}\sphinxstyleliteralemphasis{\sphinxupquote{(}}\sphinxstyleliteralemphasis{\sphinxupquote{tuple}}\sphinxstyleliteralemphasis{\sphinxupquote{ or }}\sphinxstyleliteralemphasis{\sphinxupquote{list of datetime.datetime}}\sphinxstyleliteralemphasis{\sphinxupquote{ or }}\sphinxstyleliteralemphasis{\sphinxupquote{None}}\sphinxstyleliteralemphasis{\sphinxupquote{)}}\sphinxstyleliteralemphasis{\sphinxupquote{, }}\sphinxstyleliteralemphasis{\sphinxupquote{optional}}) \textendash{} The minimum and maximum Timestamp for which to get the timeseries.
If None is given, the maximum or minimal possible Timestamp is taken.
The default is (None, None).

\item {} 
\sphinxAtStartPar
\sphinxstyleliteralstrong{\sphinxupquote{agg\_to}} (\sphinxstyleliteralemphasis{\sphinxupquote{str}}\sphinxstyleliteralemphasis{\sphinxupquote{ or }}\sphinxstyleliteralemphasis{\sphinxupquote{None}}\sphinxstyleliteralemphasis{\sphinxupquote{, }}\sphinxstyleliteralemphasis{\sphinxupquote{optional}}) \textendash{} Aggregate to a given timespan.
Can be anything smaller than the maximum timespan of the saved data.
If a Timeperiod smaller than the saved data is given, than the maximum possible timeperiod is returned.
For T and ET it can be “month”, “year”.
For N it can also be “hour”.
If None than the maximum timeperiod is taken.
The default is None.

\end{itemize}

\item[{Returns}] \leavevmode
\sphinxAtStartPar
A timeserie with the adjusted data.

\item[{Return type}] \leavevmode
\sphinxAtStartPar
pandas.DataFrame

\end{description}\end{quote}

\end{fulllineitems}

\index{isin\_meta\_n() (weatherDB.station.StationTETBase method)@\spxentry{isin\_meta\_n()}\spxextra{weatherDB.station.StationTETBase method}}

\begin{fulllineitems}
\phantomsection\label{\detokenize{weatherDB:weatherDB.station.StationTETBase.isin_meta_n}}\pysiglinewithargsret{\sphinxbfcode{\sphinxupquote{isin\_meta\_n}}}{}{}
\sphinxAtStartPar
Check if Station is in the precipitation meta table.
\begin{quote}\begin{description}
\item[{Returns}] \leavevmode
\sphinxAtStartPar
True if Station is in the precipitation meta table.

\item[{Return type}] \leavevmode
\sphinxAtStartPar
bool

\end{description}\end{quote}

\end{fulllineitems}

\index{quality\_check() (weatherDB.station.StationTETBase method)@\spxentry{quality\_check()}\spxextra{weatherDB.station.StationTETBase method}}

\begin{fulllineitems}
\phantomsection\label{\detokenize{weatherDB:weatherDB.station.StationTETBase.quality_check}}\pysiglinewithargsret{\sphinxbfcode{\sphinxupquote{quality\_check}}}{\emph{\DUrole{n}{period}\DUrole{o}{=}\DUrole{default_value}{(None, None)}}}{}
\sphinxAtStartPar
Quality check the raw data for a given period.
\begin{quote}\begin{description}
\item[{Parameters}] \leavevmode
\sphinxAtStartPar
\sphinxstyleliteralstrong{\sphinxupquote{period}} (\sphinxstyleliteralemphasis{\sphinxupquote{util.TimestampPeriod}}\sphinxstyleliteralemphasis{\sphinxupquote{ or }}\sphinxstyleliteralemphasis{\sphinxupquote{(}}\sphinxstyleliteralemphasis{\sphinxupquote{tuple}}\sphinxstyleliteralemphasis{\sphinxupquote{ or }}\sphinxstyleliteralemphasis{\sphinxupquote{list of datetime.datetime}}\sphinxstyleliteralemphasis{\sphinxupquote{ or }}\sphinxstyleliteralemphasis{\sphinxupquote{None}}\sphinxstyleliteralemphasis{\sphinxupquote{)}}\sphinxstyleliteralemphasis{\sphinxupquote{, }}\sphinxstyleliteralemphasis{\sphinxupquote{optional}}) \textendash{} The minimum and maximum Timestamp for which to get the timeseries.
If None is given, the maximum or minimal possible Timestamp is taken.
The default is (None, None).

\end{description}\end{quote}

\end{fulllineitems}


\end{fulllineitems}

\index{TemperatureStation (class in weatherDB.station)@\spxentry{TemperatureStation}\spxextra{class in weatherDB.station}}

\begin{fulllineitems}
\phantomsection\label{\detokenize{weatherDB:weatherDB.station.TemperatureStation}}\pysiglinewithargsret{\sphinxbfcode{\sphinxupquote{class\DUrole{w}{  }}}\sphinxcode{\sphinxupquote{weatherDB.station.}}\sphinxbfcode{\sphinxupquote{TemperatureStation}}}{\emph{\DUrole{n}{id}}}{}
\sphinxAtStartPar
Bases: {\hyperref[\detokenize{weatherDB:weatherDB.station.StationTETBase}]{\sphinxcrossref{\sphinxcode{\sphinxupquote{weatherDB.station.StationTETBase}}}}}
\index{\_\_init\_\_() (weatherDB.station.TemperatureStation method)@\spxentry{\_\_init\_\_()}\spxextra{weatherDB.station.TemperatureStation method}}

\begin{fulllineitems}
\phantomsection\label{\detokenize{weatherDB:weatherDB.station.TemperatureStation.__init__}}\pysiglinewithargsret{\sphinxbfcode{\sphinxupquote{\_\_init\_\_}}}{\emph{\DUrole{n}{id}}}{}
\end{fulllineitems}

\index{get\_adj() (weatherDB.station.TemperatureStation method)@\spxentry{get\_adj()}\spxextra{weatherDB.station.TemperatureStation method}}

\begin{fulllineitems}
\phantomsection\label{\detokenize{weatherDB:weatherDB.station.TemperatureStation.get_adj}}\pysiglinewithargsret{\sphinxbfcode{\sphinxupquote{get\_adj}}}{\emph{\DUrole{n}{period}\DUrole{o}{=}\DUrole{default_value}{(None, None)}}}{}
\sphinxAtStartPar
Get the adjusted timeserie.

\sphinxAtStartPar
The timeserie is adjusted to the multi annual mean.
So the overall mean of the given period will be the same as the multi annual mean.
\begin{quote}\begin{description}
\item[{Parameters}] \leavevmode\begin{itemize}
\item {} 
\sphinxAtStartPar
\sphinxstyleliteralstrong{\sphinxupquote{period}} ({\hyperref[\detokenize{weatherDB.lib:weatherDB.lib.utils.TimestampPeriod}]{\sphinxcrossref{\sphinxstyleliteralemphasis{\sphinxupquote{TimestampPeriod}}}}}\sphinxstyleliteralemphasis{\sphinxupquote{ or }}\sphinxstyleliteralemphasis{\sphinxupquote{(}}\sphinxstyleliteralemphasis{\sphinxupquote{tuple}}\sphinxstyleliteralemphasis{\sphinxupquote{ or }}\sphinxstyleliteralemphasis{\sphinxupquote{list of datetime.datetime}}\sphinxstyleliteralemphasis{\sphinxupquote{ or }}\sphinxstyleliteralemphasis{\sphinxupquote{None}}\sphinxstyleliteralemphasis{\sphinxupquote{)}}\sphinxstyleliteralemphasis{\sphinxupquote{, }}\sphinxstyleliteralemphasis{\sphinxupquote{optional}}) \textendash{} The minimum and maximum Timestamp for which to get the timeseries.
If None is given, the maximum or minimal possible Timestamp is taken.
The default is (None, None).

\item {} 
\sphinxAtStartPar
\sphinxstyleliteralstrong{\sphinxupquote{agg\_to}} (\sphinxstyleliteralemphasis{\sphinxupquote{str}}\sphinxstyleliteralemphasis{\sphinxupquote{ or }}\sphinxstyleliteralemphasis{\sphinxupquote{None}}\sphinxstyleliteralemphasis{\sphinxupquote{, }}\sphinxstyleliteralemphasis{\sphinxupquote{optional}}) \textendash{} Aggregate to a given timespan.
Can be anything smaller than the maximum timespan of the saved data.
If a Timeperiod smaller than the saved data is given, than the maximum possible timeperiod is returned.
For T and ET it can be “month”, “year”.
For N it can also be “hour”.
If None than the maximum timeperiod is taken.
The default is None.

\end{itemize}

\item[{Returns}] \leavevmode
\sphinxAtStartPar
A timeserie with the adjusted data.

\item[{Return type}] \leavevmode
\sphinxAtStartPar
pandas.DataFrame

\end{description}\end{quote}

\end{fulllineitems}

\index{get\_multi\_annual() (weatherDB.station.TemperatureStation method)@\spxentry{get\_multi\_annual()}\spxextra{weatherDB.station.TemperatureStation method}}

\begin{fulllineitems}
\phantomsection\label{\detokenize{weatherDB:weatherDB.station.TemperatureStation.get_multi_annual}}\pysiglinewithargsret{\sphinxbfcode{\sphinxupquote{get\_multi\_annual}}}{}{}
\sphinxAtStartPar
Get the multi annual value(s) for this station.
\begin{quote}\begin{description}
\item[{Returns}] \leavevmode
\sphinxAtStartPar
The corresponding multi annual value.
For T en ET the yearly value is returned.
For N the winter and summer half yearly sum is returned in tuple.

\item[{Return type}] \leavevmode
\sphinxAtStartPar
list or number

\end{description}\end{quote}

\end{fulllineitems}


\end{fulllineitems}



\subsection{weatherDB.stations module}
\label{\detokenize{weatherDB:module-weatherDB.stations}}\label{\detokenize{weatherDB:weatherdb-stations-module}}\index{module@\spxentry{module}!weatherDB.stations@\spxentry{weatherDB.stations}}\index{weatherDB.stations@\spxentry{weatherDB.stations}!module@\spxentry{module}}\index{EvapotranspirationStations (class in weatherDB.stations)@\spxentry{EvapotranspirationStations}\spxextra{class in weatherDB.stations}}

\begin{fulllineitems}
\phantomsection\label{\detokenize{weatherDB:weatherDB.stations.EvapotranspirationStations}}\pysigline{\sphinxbfcode{\sphinxupquote{class\DUrole{w}{  }}}\sphinxcode{\sphinxupquote{weatherDB.stations.}}\sphinxbfcode{\sphinxupquote{EvapotranspirationStations}}}
\sphinxAtStartPar
Bases: {\hyperref[\detokenize{weatherDB:weatherDB.stations.StationsTETBase}]{\sphinxcrossref{\sphinxcode{\sphinxupquote{weatherDB.stations.StationsTETBase}}}}}

\end{fulllineitems}

\index{GroupStations (class in weatherDB.stations)@\spxentry{GroupStations}\spxextra{class in weatherDB.stations}}

\begin{fulllineitems}
\phantomsection\label{\detokenize{weatherDB:weatherDB.stations.GroupStations}}\pysigline{\sphinxbfcode{\sphinxupquote{class\DUrole{w}{  }}}\sphinxcode{\sphinxupquote{weatherDB.stations.}}\sphinxbfcode{\sphinxupquote{GroupStations}}}
\sphinxAtStartPar
Bases: \sphinxcode{\sphinxupquote{object}}

\sphinxAtStartPar
A class to group all possible parameters of all the stations.
\index{\_\_init\_\_() (weatherDB.stations.GroupStations method)@\spxentry{\_\_init\_\_()}\spxextra{weatherDB.stations.GroupStations method}}

\begin{fulllineitems}
\phantomsection\label{\detokenize{weatherDB:weatherDB.stations.GroupStations.__init__}}\pysiglinewithargsret{\sphinxbfcode{\sphinxupquote{\_\_init\_\_}}}{}{}
\end{fulllineitems}

\index{create\_roger\_ts() (weatherDB.stations.GroupStations method)@\spxentry{create\_roger\_ts()}\spxextra{weatherDB.stations.GroupStations method}}

\begin{fulllineitems}
\phantomsection\label{\detokenize{weatherDB:weatherDB.stations.GroupStations.create_roger_ts}}\pysiglinewithargsret{\sphinxbfcode{\sphinxupquote{create\_roger\_ts}}}{\emph{\DUrole{n}{dir}}, \emph{\DUrole{n}{period}\DUrole{o}{=}\DUrole{default_value}{(None, None)}}, \emph{\DUrole{n}{kind}\DUrole{o}{=}\DUrole{default_value}{\textquotesingle{}best\textquotesingle{}}}, \emph{\DUrole{n}{stids}\DUrole{o}{=}\DUrole{default_value}{\textquotesingle{}all\textquotesingle{}}}, \emph{\DUrole{n}{zip}\DUrole{o}{=}\DUrole{default_value}{False}}}{}
\sphinxAtStartPar
Create the roger weather tables.
\begin{quote}\begin{description}
\item[{Parameters}] \leavevmode\begin{itemize}
\item {} 
\sphinxAtStartPar
\sphinxstyleliteralstrong{\sphinxupquote{dir}} (\sphinxstyleliteralemphasis{\sphinxupquote{path\sphinxhyphen{}like object}}) \textendash{} The directory where to save the tables.

\item {} 
\sphinxAtStartPar
\sphinxstyleliteralstrong{\sphinxupquote{period}} (\sphinxstyleliteralemphasis{\sphinxupquote{tuple}}\sphinxstyleliteralemphasis{\sphinxupquote{ or }}{\hyperref[\detokenize{weatherDB.lib:weatherDB.lib.utils.TimestampPeriod}]{\sphinxcrossref{\sphinxstyleliteralemphasis{\sphinxupquote{TimestampPeriod}}}}}\sphinxstyleliteralemphasis{\sphinxupquote{, }}\sphinxstyleliteralemphasis{\sphinxupquote{optional}}) \textendash{} , by default (None, None)

\item {} 
\sphinxAtStartPar
\sphinxstyleliteralstrong{\sphinxupquote{kind}} (\sphinxstyleliteralemphasis{\sphinxupquote{str}}\sphinxstyleliteralemphasis{\sphinxupquote{, }}\sphinxstyleliteralemphasis{\sphinxupquote{optional}}) \textendash{} \_description\_, by default “best”

\item {} 
\sphinxAtStartPar
\sphinxstyleliteralstrong{\sphinxupquote{stids}} (\sphinxstyleliteralemphasis{\sphinxupquote{string}}\sphinxstyleliteralemphasis{\sphinxupquote{ or }}\sphinxstyleliteralemphasis{\sphinxupquote{list of int}}\sphinxstyleliteralemphasis{\sphinxupquote{, }}\sphinxstyleliteralemphasis{\sphinxupquote{optional}}) \textendash{} The Stations for which to compute.
Can either be “all”, for all possible stations
or a list with the Station IDs.
The default is “all”.

\item {} 
\sphinxAtStartPar
\sphinxstyleliteralstrong{\sphinxupquote{zip}} (\sphinxstyleliteralemphasis{\sphinxupquote{bool}}\sphinxstyleliteralemphasis{\sphinxupquote{, }}\sphinxstyleliteralemphasis{\sphinxupquote{optional}}) \textendash{} Should the outcome get zipped in one arcive.
The default is False.

\end{itemize}

\end{description}\end{quote}

\end{fulllineitems}

\index{get\_meta() (weatherDB.stations.GroupStations method)@\spxentry{get\_meta()}\spxextra{weatherDB.stations.GroupStations method}}

\begin{fulllineitems}
\phantomsection\label{\detokenize{weatherDB:weatherDB.stations.GroupStations.get_meta}}\pysiglinewithargsret{\sphinxbfcode{\sphinxupquote{get\_meta}}}{\emph{\DUrole{n}{columns}\DUrole{o}{=}\DUrole{default_value}{{[}\textquotesingle{}Station\_id\textquotesingle{}, \textquotesingle{}von\_datum\textquotesingle{}, \textquotesingle{}bis\_datum\textquotesingle{}, \textquotesingle{}geometry\textquotesingle{}{]}}}}{}
\sphinxAtStartPar
Get the meta Dataframe from the Database.
\begin{quote}\begin{description}
\item[{Parameters}] \leavevmode
\sphinxAtStartPar
\sphinxstyleliteralstrong{\sphinxupquote{columns}} (\sphinxstyleliteralemphasis{\sphinxupquote{list}}\sphinxstyleliteralemphasis{\sphinxupquote{, }}\sphinxstyleliteralemphasis{\sphinxupquote{optional}}) \textendash{} A list of columns from the meta file to return
The default is: {[}“Station\_id”, “von\_datum”, “bis\_datum”, “geometry”{]}

\item[{Returns}] \leavevmode
\sphinxAtStartPar
The meta DataFrame.

\item[{Return type}] \leavevmode
\sphinxAtStartPar
pandas.DataFrame or geopandas.GeoDataFrae

\end{description}\end{quote}

\end{fulllineitems}

\index{get\_stations() (weatherDB.stations.GroupStations method)@\spxentry{get\_stations()}\spxextra{weatherDB.stations.GroupStations method}}

\begin{fulllineitems}
\phantomsection\label{\detokenize{weatherDB:weatherDB.stations.GroupStations.get_stations}}\pysiglinewithargsret{\sphinxbfcode{\sphinxupquote{get\_stations}}}{\emph{\DUrole{n}{stids}\DUrole{o}{=}\DUrole{default_value}{\textquotesingle{}all\textquotesingle{}}}}{}
\sphinxAtStartPar
Get a list with all the stations as Station\sphinxhyphen{}objects.
\begin{quote}\begin{description}
\item[{Parameters}] \leavevmode
\sphinxAtStartPar
\sphinxstyleliteralstrong{\sphinxupquote{stids}} (\sphinxstyleliteralemphasis{\sphinxupquote{string}}\sphinxstyleliteralemphasis{\sphinxupquote{ or }}\sphinxstyleliteralemphasis{\sphinxupquote{list of int}}\sphinxstyleliteralemphasis{\sphinxupquote{, }}\sphinxstyleliteralemphasis{\sphinxupquote{optional}}) \textendash{} The Stations to return.
Can either be “all”, for all possible stations
or a list with the Station IDs.
The default is “all”.

\item[{Returns}] \leavevmode
\sphinxAtStartPar
returns a list with the corresponding station objects.

\item[{Return type}] \leavevmode
\sphinxAtStartPar
Station\sphinxhyphen{}object

\item[{Raises}] \leavevmode
\sphinxAtStartPar
\sphinxstyleliteralstrong{\sphinxupquote{ValueError}} \textendash{} If the given stids (Station\_IDs) are not all valid.

\end{description}\end{quote}

\end{fulllineitems}


\end{fulllineitems}

\index{PrecipitationDailyStations (class in weatherDB.stations)@\spxentry{PrecipitationDailyStations}\spxextra{class in weatherDB.stations}}

\begin{fulllineitems}
\phantomsection\label{\detokenize{weatherDB:weatherDB.stations.PrecipitationDailyStations}}\pysigline{\sphinxbfcode{\sphinxupquote{class\DUrole{w}{  }}}\sphinxcode{\sphinxupquote{weatherDB.stations.}}\sphinxbfcode{\sphinxupquote{PrecipitationDailyStations}}}
\sphinxAtStartPar
Bases: {\hyperref[\detokenize{weatherDB:weatherDB.stations.StationsBase}]{\sphinxcrossref{\sphinxcode{\sphinxupquote{weatherDB.stations.StationsBase}}}}}

\end{fulllineitems}

\index{PrecipitationStations (class in weatherDB.stations)@\spxentry{PrecipitationStations}\spxextra{class in weatherDB.stations}}

\begin{fulllineitems}
\phantomsection\label{\detokenize{weatherDB:weatherDB.stations.PrecipitationStations}}\pysigline{\sphinxbfcode{\sphinxupquote{class\DUrole{w}{  }}}\sphinxcode{\sphinxupquote{weatherDB.stations.}}\sphinxbfcode{\sphinxupquote{PrecipitationStations}}}
\sphinxAtStartPar
Bases: {\hyperref[\detokenize{weatherDB:weatherDB.stations.StationsBase}]{\sphinxcrossref{\sphinxcode{\sphinxupquote{weatherDB.stations.StationsBase}}}}}
\index{last\_imp\_corr() (weatherDB.stations.PrecipitationStations method)@\spxentry{last\_imp\_corr()}\spxextra{weatherDB.stations.PrecipitationStations method}}

\begin{fulllineitems}
\phantomsection\label{\detokenize{weatherDB:weatherDB.stations.PrecipitationStations.last_imp_corr}}\pysiglinewithargsret{\sphinxbfcode{\sphinxupquote{last\_imp\_corr}}}{\emph{\DUrole{n}{stids}\DUrole{o}{=}\DUrole{default_value}{\textquotesingle{}all\textquotesingle{}}}}{}
\sphinxAtStartPar
Richter correct the filled data for the last imported period.
\begin{quote}\begin{description}
\item[{Parameters}] \leavevmode
\sphinxAtStartPar
\sphinxstyleliteralstrong{\sphinxupquote{stids}} (\sphinxstyleliteralemphasis{\sphinxupquote{string}}\sphinxstyleliteralemphasis{\sphinxupquote{ or }}\sphinxstyleliteralemphasis{\sphinxupquote{list of int}}\sphinxstyleliteralemphasis{\sphinxupquote{, }}\sphinxstyleliteralemphasis{\sphinxupquote{optional}}) \textendash{} The Stations for which to compute.
Can either be “all”, for all possible stations
or a list with the Station IDs.
The default is “all”.

\item[{Raises}] \leavevmode
\sphinxAtStartPar
\sphinxstyleliteralstrong{\sphinxupquote{ValueError}} \textendash{} If the given stids (Station\_IDs) are not all valid.

\end{description}\end{quote}

\end{fulllineitems}

\index{richter\_correct() (weatherDB.stations.PrecipitationStations method)@\spxentry{richter\_correct()}\spxextra{weatherDB.stations.PrecipitationStations method}}

\begin{fulllineitems}
\phantomsection\label{\detokenize{weatherDB:weatherDB.stations.PrecipitationStations.richter_correct}}\pysiglinewithargsret{\sphinxbfcode{\sphinxupquote{richter\_correct}}}{\emph{\DUrole{n}{stids}\DUrole{o}{=}\DUrole{default_value}{\textquotesingle{}all\textquotesingle{}}}}{}
\sphinxAtStartPar
Richter correct the filled data.
\begin{quote}\begin{description}
\item[{Parameters}] \leavevmode
\sphinxAtStartPar
\sphinxstyleliteralstrong{\sphinxupquote{stids}} (\sphinxstyleliteralemphasis{\sphinxupquote{string}}\sphinxstyleliteralemphasis{\sphinxupquote{ or }}\sphinxstyleliteralemphasis{\sphinxupquote{list of int}}\sphinxstyleliteralemphasis{\sphinxupquote{, }}\sphinxstyleliteralemphasis{\sphinxupquote{optional}}) \textendash{} The Stations for which to compute.
Can either be “all”, for all possible stations
or a list with the Station IDs.
The default is “all”.

\item[{Raises}] \leavevmode
\sphinxAtStartPar
\sphinxstyleliteralstrong{\sphinxupquote{ValueError}} \textendash{} If the given stids (Station\_IDs) are not all valid.

\end{description}\end{quote}

\end{fulllineitems}

\index{update\_richter\_class() (weatherDB.stations.PrecipitationStations method)@\spxentry{update\_richter\_class()}\spxextra{weatherDB.stations.PrecipitationStations method}}

\begin{fulllineitems}
\phantomsection\label{\detokenize{weatherDB:weatherDB.stations.PrecipitationStations.update_richter_class}}\pysiglinewithargsret{\sphinxbfcode{\sphinxupquote{update\_richter\_class}}}{\emph{\DUrole{n}{stids}\DUrole{o}{=}\DUrole{default_value}{\textquotesingle{}all\textquotesingle{}}}}{}
\sphinxAtStartPar
Update the Richter exposition class.

\sphinxAtStartPar
Get the value from the raster, compare with the richter categories and save to the database.
\begin{quote}\begin{description}
\item[{Parameters}] \leavevmode
\sphinxAtStartPar
\sphinxstyleliteralstrong{\sphinxupquote{stids}} (\sphinxstyleliteralemphasis{\sphinxupquote{string}}\sphinxstyleliteralemphasis{\sphinxupquote{ or }}\sphinxstyleliteralemphasis{\sphinxupquote{list of int}}\sphinxstyleliteralemphasis{\sphinxupquote{, }}\sphinxstyleliteralemphasis{\sphinxupquote{optional}}) \textendash{} The Stations for which to compute.
Can either be “all”, for all possible stations
or a list with the Station IDs.
The default is “all”.

\item[{Raises}] \leavevmode
\sphinxAtStartPar
\sphinxstyleliteralstrong{\sphinxupquote{ValueError}} \textendash{} If the given stids (Station\_IDs) are not all valid.

\end{description}\end{quote}

\end{fulllineitems}


\end{fulllineitems}

\index{StationsBase (class in weatherDB.stations)@\spxentry{StationsBase}\spxextra{class in weatherDB.stations}}

\begin{fulllineitems}
\phantomsection\label{\detokenize{weatherDB:weatherDB.stations.StationsBase}}\pysigline{\sphinxbfcode{\sphinxupquote{class\DUrole{w}{  }}}\sphinxcode{\sphinxupquote{weatherDB.stations.}}\sphinxbfcode{\sphinxupquote{StationsBase}}}
\sphinxAtStartPar
Bases: \sphinxcode{\sphinxupquote{object}}
\index{\_\_init\_\_() (weatherDB.stations.StationsBase method)@\spxentry{\_\_init\_\_()}\spxextra{weatherDB.stations.StationsBase method}}

\begin{fulllineitems}
\phantomsection\label{\detokenize{weatherDB:weatherDB.stations.StationsBase.__init__}}\pysiglinewithargsret{\sphinxbfcode{\sphinxupquote{\_\_init\_\_}}}{}{}
\end{fulllineitems}

\index{download\_meta() (weatherDB.stations.StationsBase method)@\spxentry{download\_meta()}\spxextra{weatherDB.stations.StationsBase method}}

\begin{fulllineitems}
\phantomsection\label{\detokenize{weatherDB:weatherDB.stations.StationsBase.download_meta}}\pysiglinewithargsret{\sphinxbfcode{\sphinxupquote{download\_meta}}}{}{}
\sphinxAtStartPar
Download the meta file(s) from the CDC server.
\begin{quote}\begin{description}
\item[{Returns}] \leavevmode
\sphinxAtStartPar
The meta file from the CDC server.
If there are several meta files on the server, they are joined together.

\item[{Return type}] \leavevmode
\sphinxAtStartPar
geopandas.GeoDataFrame

\end{description}\end{quote}

\end{fulllineitems}

\index{fillup() (weatherDB.stations.StationsBase method)@\spxentry{fillup()}\spxextra{weatherDB.stations.StationsBase method}}

\begin{fulllineitems}
\phantomsection\label{\detokenize{weatherDB:weatherDB.stations.StationsBase.fillup}}\pysiglinewithargsret{\sphinxbfcode{\sphinxupquote{fillup}}}{\emph{\DUrole{n}{only\_real}\DUrole{o}{=}\DUrole{default_value}{False}}, \emph{\DUrole{n}{stids}\DUrole{o}{=}\DUrole{default_value}{\textquotesingle{}all\textquotesingle{}}}}{}
\sphinxAtStartPar
Fill up the quality checked data with data from nearby stations to get complete timeseries.
\begin{quote}\begin{description}
\item[{Parameters}] \leavevmode\begin{itemize}
\item {} 
\sphinxAtStartPar
\sphinxstyleliteralstrong{\sphinxupquote{only\_real}} (\sphinxstyleliteralemphasis{\sphinxupquote{bool}}\sphinxstyleliteralemphasis{\sphinxupquote{, }}\sphinxstyleliteralemphasis{\sphinxupquote{optional}}) \textendash{} Whether only real stations are computed or also virtual ones.
True: only stations with own data are returned.
The default is True.

\item {} 
\sphinxAtStartPar
\sphinxstyleliteralstrong{\sphinxupquote{stids}} (\sphinxstyleliteralemphasis{\sphinxupquote{string}}\sphinxstyleliteralemphasis{\sphinxupquote{ or }}\sphinxstyleliteralemphasis{\sphinxupquote{list of int}}\sphinxstyleliteralemphasis{\sphinxupquote{, }}\sphinxstyleliteralemphasis{\sphinxupquote{optional}}) \textendash{} The Stations for which to compute.
Can either be “all”, for all possible stations
or a list with the Station IDs.
The default is “all”.

\end{itemize}

\item[{Raises}] \leavevmode
\sphinxAtStartPar
\sphinxstyleliteralstrong{\sphinxupquote{ValueError}} \textendash{} If the given stids (Station\_IDs) are not all valid.

\end{description}\end{quote}

\end{fulllineitems}

\index{get\_meta() (weatherDB.stations.StationsBase method)@\spxentry{get\_meta()}\spxextra{weatherDB.stations.StationsBase method}}

\begin{fulllineitems}
\phantomsection\label{\detokenize{weatherDB:weatherDB.stations.StationsBase.get_meta}}\pysiglinewithargsret{\sphinxbfcode{\sphinxupquote{get\_meta}}}{\emph{\DUrole{n}{columns}\DUrole{o}{=}\DUrole{default_value}{{[}\textquotesingle{}Station\_id\textquotesingle{}, \textquotesingle{}von\_datum\textquotesingle{}, \textquotesingle{}bis\_datum\textquotesingle{}, \textquotesingle{}geometry\textquotesingle{}{]}}}, \emph{\DUrole{n}{only\_real}\DUrole{o}{=}\DUrole{default_value}{True}}}{}
\sphinxAtStartPar
Get the meta Dataframe from the Database.
\begin{quote}\begin{description}
\item[{Parameters}] \leavevmode\begin{itemize}
\item {} 
\sphinxAtStartPar
\sphinxstyleliteralstrong{\sphinxupquote{columns}} (\sphinxstyleliteralemphasis{\sphinxupquote{list}}\sphinxstyleliteralemphasis{\sphinxupquote{, }}\sphinxstyleliteralemphasis{\sphinxupquote{optional}}) \textendash{} A list of columns from the meta file to return
The default is: {[}“Station\_id”, “von\_datum”, “bis\_datum”, “geometry”{]}

\item {} 
\sphinxAtStartPar
\sphinxstyleliteralstrong{\sphinxupquote{only\_real}} (\sphinxstyleliteralemphasis{\sphinxupquote{bool}}\sphinxstyleliteralemphasis{\sphinxupquote{, }}\sphinxstyleliteralemphasis{\sphinxupquote{optional}}) \textendash{} Whether only real stations are returned or also virtual ones.
True: only stations with own data are returned.
The default is True.

\end{itemize}

\item[{Returns}] \leavevmode
\sphinxAtStartPar
The meta DataFrame.

\item[{Return type}] \leavevmode
\sphinxAtStartPar
pandas.DataFrame or geopandas.GeoDataFrae

\end{description}\end{quote}

\end{fulllineitems}

\index{get\_stations() (weatherDB.stations.StationsBase method)@\spxentry{get\_stations()}\spxextra{weatherDB.stations.StationsBase method}}

\begin{fulllineitems}
\phantomsection\label{\detokenize{weatherDB:weatherDB.stations.StationsBase.get_stations}}\pysiglinewithargsret{\sphinxbfcode{\sphinxupquote{get\_stations}}}{\emph{\DUrole{n}{only\_real}\DUrole{o}{=}\DUrole{default_value}{True}}, \emph{\DUrole{n}{stids}\DUrole{o}{=}\DUrole{default_value}{\textquotesingle{}all\textquotesingle{}}}}{}
\sphinxAtStartPar
Get a list with all the stations as Station\sphinxhyphen{}objects.
\begin{quote}\begin{description}
\item[{Parameters}] \leavevmode\begin{itemize}
\item {} 
\sphinxAtStartPar
\sphinxstyleliteralstrong{\sphinxupquote{only\_real}} (\sphinxstyleliteralemphasis{\sphinxupquote{bool}}\sphinxstyleliteralemphasis{\sphinxupquote{, }}\sphinxstyleliteralemphasis{\sphinxupquote{optional}}) \textendash{} Whether only real stations are returned or also virtual ones.
True: only stations with own data are returned.
The default is True.

\item {} 
\sphinxAtStartPar
\sphinxstyleliteralstrong{\sphinxupquote{stids}} (\sphinxstyleliteralemphasis{\sphinxupquote{string}}\sphinxstyleliteralemphasis{\sphinxupquote{ or }}\sphinxstyleliteralemphasis{\sphinxupquote{list of int}}\sphinxstyleliteralemphasis{\sphinxupquote{, }}\sphinxstyleliteralemphasis{\sphinxupquote{optional}}) \textendash{} The Stations to return.
Can either be “all”, for all possible stations
or a list with the Station IDs.
The default is “all”.

\end{itemize}

\item[{Returns}] \leavevmode
\sphinxAtStartPar
returns a list with the corresponding station objects.

\item[{Return type}] \leavevmode
\sphinxAtStartPar
Station\sphinxhyphen{}object

\item[{Raises}] \leavevmode
\sphinxAtStartPar
\sphinxstyleliteralstrong{\sphinxupquote{ValueError}} \textendash{} If the given stids (Station\_IDs) are not all valid.

\end{description}\end{quote}

\end{fulllineitems}

\index{last\_imp\_fillup() (weatherDB.stations.StationsBase method)@\spxentry{last\_imp\_fillup()}\spxextra{weatherDB.stations.StationsBase method}}

\begin{fulllineitems}
\phantomsection\label{\detokenize{weatherDB:weatherDB.stations.StationsBase.last_imp_fillup}}\pysiglinewithargsret{\sphinxbfcode{\sphinxupquote{last\_imp\_fillup}}}{}{}
\sphinxAtStartPar
Do the filling of the last import.

\end{fulllineitems}

\index{last\_imp\_quality\_check() (weatherDB.stations.StationsBase method)@\spxentry{last\_imp\_quality\_check()}\spxextra{weatherDB.stations.StationsBase method}}

\begin{fulllineitems}
\phantomsection\label{\detokenize{weatherDB:weatherDB.stations.StationsBase.last_imp_quality_check}}\pysiglinewithargsret{\sphinxbfcode{\sphinxupquote{last\_imp\_quality\_check}}}{}{}
\sphinxAtStartPar
Do the quality check of the last import.

\end{fulllineitems}

\index{quality\_check() (weatherDB.stations.StationsBase method)@\spxentry{quality\_check()}\spxextra{weatherDB.stations.StationsBase method}}

\begin{fulllineitems}
\phantomsection\label{\detokenize{weatherDB:weatherDB.stations.StationsBase.quality_check}}\pysiglinewithargsret{\sphinxbfcode{\sphinxupquote{quality\_check}}}{\emph{\DUrole{n}{period}\DUrole{o}{=}\DUrole{default_value}{(None, None)}}, \emph{\DUrole{n}{only\_real}\DUrole{o}{=}\DUrole{default_value}{True}}, \emph{\DUrole{n}{stids}\DUrole{o}{=}\DUrole{default_value}{\textquotesingle{}all\textquotesingle{}}}}{}
\sphinxAtStartPar
Quality check the raw data for a given period.
\begin{quote}\begin{description}
\item[{Parameters}] \leavevmode\begin{itemize}
\item {} 
\sphinxAtStartPar
\sphinxstyleliteralstrong{\sphinxupquote{period}} (\sphinxstyleliteralemphasis{\sphinxupquote{tuple}}\sphinxstyleliteralemphasis{\sphinxupquote{ or }}\sphinxstyleliteralemphasis{\sphinxupquote{list of datetime.datetime}}\sphinxstyleliteralemphasis{\sphinxupquote{ or }}\sphinxstyleliteralemphasis{\sphinxupquote{None}}\sphinxstyleliteralemphasis{\sphinxupquote{, }}\sphinxstyleliteralemphasis{\sphinxupquote{optional}}) \textendash{} The minimum and maximum Timestamp for which to get the timeseries.
If None is given, the maximum or minimal possible Timestamp is taken.
The default is (None, None).

\item {} 
\sphinxAtStartPar
\sphinxstyleliteralstrong{\sphinxupquote{stids}} (\sphinxstyleliteralemphasis{\sphinxupquote{string}}\sphinxstyleliteralemphasis{\sphinxupquote{ or }}\sphinxstyleliteralemphasis{\sphinxupquote{list of int}}\sphinxstyleliteralemphasis{\sphinxupquote{, }}\sphinxstyleliteralemphasis{\sphinxupquote{optional}}) \textendash{} The Stations for which to compute.
Can either be “all”, for all possible stations
or a list with the Station IDs.
The default is “all”.

\end{itemize}

\end{description}\end{quote}

\end{fulllineitems}

\index{update\_ma() (weatherDB.stations.StationsBase method)@\spxentry{update\_ma()}\spxextra{weatherDB.stations.StationsBase method}}

\begin{fulllineitems}
\phantomsection\label{\detokenize{weatherDB:weatherDB.stations.StationsBase.update_ma}}\pysiglinewithargsret{\sphinxbfcode{\sphinxupquote{update\_ma}}}{\emph{\DUrole{n}{stids}\DUrole{o}{=}\DUrole{default_value}{\textquotesingle{}all\textquotesingle{}}}}{}
\sphinxAtStartPar
Update the multi annual values for the stations.

\sphinxAtStartPar
Get a multi annual value from the corresponding raster and save to the multi annual table in the database.
\begin{quote}\begin{description}
\item[{Parameters}] \leavevmode
\sphinxAtStartPar
\sphinxstyleliteralstrong{\sphinxupquote{stids}} (\sphinxstyleliteralemphasis{\sphinxupquote{string}}\sphinxstyleliteralemphasis{\sphinxupquote{ or }}\sphinxstyleliteralemphasis{\sphinxupquote{list of int}}\sphinxstyleliteralemphasis{\sphinxupquote{, }}\sphinxstyleliteralemphasis{\sphinxupquote{optional}}) \textendash{} The Stations for which to compute.
Can either be “all”, for all possible stations
or a list with the Station IDs.
The default is “all”.

\item[{Raises}] \leavevmode
\sphinxAtStartPar
\sphinxstyleliteralstrong{\sphinxupquote{ValueError}} \textendash{} If the given stids (Station\_IDs) are not all valid.

\end{description}\end{quote}

\end{fulllineitems}

\index{update\_meta() (weatherDB.stations.StationsBase method)@\spxentry{update\_meta()}\spxextra{weatherDB.stations.StationsBase method}}

\begin{fulllineitems}
\phantomsection\label{\detokenize{weatherDB:weatherDB.stations.StationsBase.update_meta}}\pysiglinewithargsret{\sphinxbfcode{\sphinxupquote{update\_meta}}}{}{}
\sphinxAtStartPar
Update the meta table by comparing to the CDC server.

\sphinxAtStartPar
The “von\_datum” and “bis\_datum” is ignored because it is better to set this by the filled period of the stations in the database.
Often the CDC period is not correct.

\end{fulllineitems}

\index{update\_period\_meta() (weatherDB.stations.StationsBase method)@\spxentry{update\_period\_meta()}\spxextra{weatherDB.stations.StationsBase method}}

\begin{fulllineitems}
\phantomsection\label{\detokenize{weatherDB:weatherDB.stations.StationsBase.update_period_meta}}\pysiglinewithargsret{\sphinxbfcode{\sphinxupquote{update\_period\_meta}}}{\emph{\DUrole{n}{stids}\DUrole{o}{=}\DUrole{default_value}{\textquotesingle{}all\textquotesingle{}}}}{}
\sphinxAtStartPar
Update the period in the meta table of the raw data.
\begin{quote}\begin{description}
\item[{Parameters}] \leavevmode
\sphinxAtStartPar
\sphinxstyleliteralstrong{\sphinxupquote{stids}} (\sphinxstyleliteralemphasis{\sphinxupquote{string}}\sphinxstyleliteralemphasis{\sphinxupquote{  or }}\sphinxstyleliteralemphasis{\sphinxupquote{list of int}}\sphinxstyleliteralemphasis{\sphinxupquote{, }}\sphinxstyleliteralemphasis{\sphinxupquote{optional}}) \textendash{} The Stations for which to compute.
Can either be “all”, for all possible stations
or a list with the Station IDs.
The default is “all”.

\item[{Raises}] \leavevmode
\sphinxAtStartPar
\sphinxstyleliteralstrong{\sphinxupquote{ValueError}} \textendash{} If the given stids (Station\_IDs) are not all valid.

\end{description}\end{quote}

\end{fulllineitems}

\index{update\_raw() (weatherDB.stations.StationsBase method)@\spxentry{update\_raw()}\spxextra{weatherDB.stations.StationsBase method}}

\begin{fulllineitems}
\phantomsection\label{\detokenize{weatherDB:weatherDB.stations.StationsBase.update_raw}}\pysiglinewithargsret{\sphinxbfcode{\sphinxupquote{update\_raw}}}{\emph{\DUrole{n}{only\_new}\DUrole{o}{=}\DUrole{default_value}{True}}, \emph{\DUrole{n}{only\_real}\DUrole{o}{=}\DUrole{default_value}{True}}, \emph{\DUrole{n}{stids}\DUrole{o}{=}\DUrole{default_value}{\textquotesingle{}all\textquotesingle{}}}}{}
\sphinxAtStartPar
Download all stations data from CDC and upload to database.
\begin{quote}\begin{description}
\item[{Parameters}] \leavevmode\begin{itemize}
\item {} 
\sphinxAtStartPar
\sphinxstyleliteralstrong{\sphinxupquote{only\_new}} (\sphinxstyleliteralemphasis{\sphinxupquote{bool}}\sphinxstyleliteralemphasis{\sphinxupquote{, }}\sphinxstyleliteralemphasis{\sphinxupquote{optional}}) \textendash{} Get only the files that are not yet in the database?
If False all the available files are loaded again.
The default is True

\item {} 
\sphinxAtStartPar
\sphinxstyleliteralstrong{\sphinxupquote{only\_real}} (\sphinxstyleliteralemphasis{\sphinxupquote{bool}}\sphinxstyleliteralemphasis{\sphinxupquote{, }}\sphinxstyleliteralemphasis{\sphinxupquote{optional}}) \textendash{} Whether only real stations are tried to download.
True: only stations with a date in von\_datum in meta are downloaded.
The default is True.

\item {} 
\sphinxAtStartPar
\sphinxstyleliteralstrong{\sphinxupquote{stids}} (\sphinxstyleliteralemphasis{\sphinxupquote{string}}\sphinxstyleliteralemphasis{\sphinxupquote{ or }}\sphinxstyleliteralemphasis{\sphinxupquote{list of int}}\sphinxstyleliteralemphasis{\sphinxupquote{, }}\sphinxstyleliteralemphasis{\sphinxupquote{optional}}) \textendash{} The Stations to return.
Can either be “all”, for all possible stations
or a list with the Station IDs.
The default is “all”.

\end{itemize}

\item[{Raises}] \leavevmode
\sphinxAtStartPar
\sphinxstyleliteralstrong{\sphinxupquote{ValueError}} \textendash{} If the given stids (Station\_IDs) are not all valid.

\end{description}\end{quote}

\end{fulllineitems}


\end{fulllineitems}

\index{StationsET (in module weatherDB.stations)@\spxentry{StationsET}\spxextra{in module weatherDB.stations}}

\begin{fulllineitems}
\phantomsection\label{\detokenize{weatherDB:weatherDB.stations.StationsET}}\pysigline{\sphinxcode{\sphinxupquote{weatherDB.stations.}}\sphinxbfcode{\sphinxupquote{StationsET}}}
\sphinxAtStartPar
alias of {\hyperref[\detokenize{weatherDB:weatherDB.stations.EvapotranspirationStations}]{\sphinxcrossref{\sphinxcode{\sphinxupquote{weatherDB.stations.EvapotranspirationStations}}}}}

\end{fulllineitems}

\index{StationsN (in module weatherDB.stations)@\spxentry{StationsN}\spxextra{in module weatherDB.stations}}

\begin{fulllineitems}
\phantomsection\label{\detokenize{weatherDB:weatherDB.stations.StationsN}}\pysigline{\sphinxcode{\sphinxupquote{weatherDB.stations.}}\sphinxbfcode{\sphinxupquote{StationsN}}}
\sphinxAtStartPar
alias of {\hyperref[\detokenize{weatherDB:weatherDB.stations.PrecipitationStations}]{\sphinxcrossref{\sphinxcode{\sphinxupquote{weatherDB.stations.PrecipitationStations}}}}}

\end{fulllineitems}

\index{StationsND (in module weatherDB.stations)@\spxentry{StationsND}\spxextra{in module weatherDB.stations}}

\begin{fulllineitems}
\phantomsection\label{\detokenize{weatherDB:weatherDB.stations.StationsND}}\pysigline{\sphinxcode{\sphinxupquote{weatherDB.stations.}}\sphinxbfcode{\sphinxupquote{StationsND}}}
\sphinxAtStartPar
alias of {\hyperref[\detokenize{weatherDB:weatherDB.stations.PrecipitationDailyStations}]{\sphinxcrossref{\sphinxcode{\sphinxupquote{weatherDB.stations.PrecipitationDailyStations}}}}}

\end{fulllineitems}

\index{StationsT (in module weatherDB.stations)@\spxentry{StationsT}\spxextra{in module weatherDB.stations}}

\begin{fulllineitems}
\phantomsection\label{\detokenize{weatherDB:weatherDB.stations.StationsT}}\pysigline{\sphinxcode{\sphinxupquote{weatherDB.stations.}}\sphinxbfcode{\sphinxupquote{StationsT}}}
\sphinxAtStartPar
alias of {\hyperref[\detokenize{weatherDB:weatherDB.stations.TemperatureStations}]{\sphinxcrossref{\sphinxcode{\sphinxupquote{weatherDB.stations.TemperatureStations}}}}}

\end{fulllineitems}

\index{StationsTETBase (class in weatherDB.stations)@\spxentry{StationsTETBase}\spxextra{class in weatherDB.stations}}

\begin{fulllineitems}
\phantomsection\label{\detokenize{weatherDB:weatherDB.stations.StationsTETBase}}\pysigline{\sphinxbfcode{\sphinxupquote{class\DUrole{w}{  }}}\sphinxcode{\sphinxupquote{weatherDB.stations.}}\sphinxbfcode{\sphinxupquote{StationsTETBase}}}
\sphinxAtStartPar
Bases: {\hyperref[\detokenize{weatherDB:weatherDB.stations.StationsBase}]{\sphinxcrossref{\sphinxcode{\sphinxupquote{weatherDB.stations.StationsBase}}}}}
\index{fillup() (weatherDB.stations.StationsTETBase method)@\spxentry{fillup()}\spxextra{weatherDB.stations.StationsTETBase method}}

\begin{fulllineitems}
\phantomsection\label{\detokenize{weatherDB:weatherDB.stations.StationsTETBase.fillup}}\pysiglinewithargsret{\sphinxbfcode{\sphinxupquote{fillup}}}{\emph{\DUrole{n}{only\_real}\DUrole{o}{=}\DUrole{default_value}{False}}, \emph{\DUrole{n}{stids}\DUrole{o}{=}\DUrole{default_value}{\textquotesingle{}all\textquotesingle{}}}}{}
\sphinxAtStartPar
Fill up the quality checked data with data from nearby stations to get complete timeseries.
\begin{quote}\begin{description}
\item[{Parameters}] \leavevmode\begin{itemize}
\item {} 
\sphinxAtStartPar
\sphinxstyleliteralstrong{\sphinxupquote{only\_real}} (\sphinxstyleliteralemphasis{\sphinxupquote{bool}}\sphinxstyleliteralemphasis{\sphinxupquote{, }}\sphinxstyleliteralemphasis{\sphinxupquote{optional}}) \textendash{} Whether only real stations are computed or also virtual ones.
True: only stations with own data are returned.
The default is True.

\item {} 
\sphinxAtStartPar
\sphinxstyleliteralstrong{\sphinxupquote{stids}} (\sphinxstyleliteralemphasis{\sphinxupquote{string}}\sphinxstyleliteralemphasis{\sphinxupquote{ or }}\sphinxstyleliteralemphasis{\sphinxupquote{list of int}}\sphinxstyleliteralemphasis{\sphinxupquote{, }}\sphinxstyleliteralemphasis{\sphinxupquote{optional}}) \textendash{} The Stations for which to compute.
Can either be “all”, for all possible stations
or a list with the Station IDs.
The default is “all”.

\end{itemize}

\item[{Raises}] \leavevmode
\sphinxAtStartPar
\sphinxstyleliteralstrong{\sphinxupquote{ValueError}} \textendash{} If the given stids (Station\_IDs) are not all valid.

\end{description}\end{quote}

\end{fulllineitems}


\end{fulllineitems}

\index{TemperatureStations (class in weatherDB.stations)@\spxentry{TemperatureStations}\spxextra{class in weatherDB.stations}}

\begin{fulllineitems}
\phantomsection\label{\detokenize{weatherDB:weatherDB.stations.TemperatureStations}}\pysigline{\sphinxbfcode{\sphinxupquote{class\DUrole{w}{  }}}\sphinxcode{\sphinxupquote{weatherDB.stations.}}\sphinxbfcode{\sphinxupquote{TemperatureStations}}}
\sphinxAtStartPar
Bases: {\hyperref[\detokenize{weatherDB:weatherDB.stations.StationsTETBase}]{\sphinxcrossref{\sphinxcode{\sphinxupquote{weatherDB.stations.StationsTETBase}}}}}

\end{fulllineitems}



\subsection{Subpackages}
\label{\detokenize{weatherDB:subpackages}}

\subsubsection{weatherDB.lib package}
\label{\detokenize{weatherDB.lib:weatherdb-lib-package}}\label{\detokenize{weatherDB.lib::doc}}

\paragraph{weatherDB.lib.connections module}
\label{\detokenize{weatherDB.lib:module-weatherDB.lib.connections}}\label{\detokenize{weatherDB.lib:weatherdb-lib-connections-module}}\index{module@\spxentry{module}!weatherDB.lib.connections@\spxentry{weatherDB.lib.connections}}\index{weatherDB.lib.connections@\spxentry{weatherDB.lib.connections}!module@\spxentry{module}}\index{FTP (class in weatherDB.lib.connections)@\spxentry{FTP}\spxextra{class in weatherDB.lib.connections}}

\begin{fulllineitems}
\phantomsection\label{\detokenize{weatherDB.lib:weatherDB.lib.connections.FTP}}\pysiglinewithargsret{\sphinxbfcode{\sphinxupquote{class\DUrole{w}{  }}}\sphinxcode{\sphinxupquote{weatherDB.lib.connections.}}\sphinxbfcode{\sphinxupquote{FTP}}}{\emph{\DUrole{n}{host=\textquotesingle{}\textquotesingle{}}}, \emph{\DUrole{n}{user=\textquotesingle{}\textquotesingle{}}}, \emph{\DUrole{n}{passwd=\textquotesingle{}\textquotesingle{}}}, \emph{\DUrole{n}{acct=\textquotesingle{}\textquotesingle{}}}, \emph{\DUrole{n}{timeout=\textless{}object object\textgreater{}}}, \emph{\DUrole{n}{source\_address=None}}, \emph{\DUrole{n}{*}}, \emph{\DUrole{n}{encoding=\textquotesingle{}utf\sphinxhyphen{}8\textquotesingle{}}}}{}
\sphinxAtStartPar
Bases: \sphinxcode{\sphinxupquote{ftplib.FTP}}
\index{login() (weatherDB.lib.connections.FTP method)@\spxentry{login()}\spxextra{weatherDB.lib.connections.FTP method}}

\begin{fulllineitems}
\phantomsection\label{\detokenize{weatherDB.lib:weatherDB.lib.connections.FTP.login}}\pysiglinewithargsret{\sphinxbfcode{\sphinxupquote{login}}}{\emph{\DUrole{o}{**}\DUrole{n}{kwargs}}}{}
\sphinxAtStartPar
Login, default anonymous.

\end{fulllineitems}


\end{fulllineitems}

\index{check\_superuser() (in module weatherDB.lib.connections)@\spxentry{check\_superuser()}\spxextra{in module weatherDB.lib.connections}}

\begin{fulllineitems}
\phantomsection\label{\detokenize{weatherDB.lib:weatherDB.lib.connections.check_superuser}}\pysiglinewithargsret{\sphinxcode{\sphinxupquote{weatherDB.lib.connections.}}\sphinxbfcode{\sphinxupquote{check\_superuser}}}{\emph{\DUrole{n}{methode}}}{}
\end{fulllineitems}



\paragraph{weatherDB.lib.utils module}
\label{\detokenize{weatherDB.lib:module-weatherDB.lib.utils}}\label{\detokenize{weatherDB.lib:weatherdb-lib-utils-module}}\index{module@\spxentry{module}!weatherDB.lib.utils@\spxentry{weatherDB.lib.utils}}\index{weatherDB.lib.utils@\spxentry{weatherDB.lib.utils}!module@\spxentry{module}}\index{TimestampPeriod (class in weatherDB.lib.utils)@\spxentry{TimestampPeriod}\spxextra{class in weatherDB.lib.utils}}

\begin{fulllineitems}
\phantomsection\label{\detokenize{weatherDB.lib:weatherDB.lib.utils.TimestampPeriod}}\pysiglinewithargsret{\sphinxbfcode{\sphinxupquote{class\DUrole{w}{  }}}\sphinxcode{\sphinxupquote{weatherDB.lib.utils.}}\sphinxbfcode{\sphinxupquote{TimestampPeriod}}}{\emph{\DUrole{n}{start}}, \emph{\DUrole{n}{end}}}{}
\sphinxAtStartPar
Bases: \sphinxcode{\sphinxupquote{object}}

\sphinxAtStartPar
A class to save a Timespan with a minimal and maximal Timestamp.
\index{COMPARE (weatherDB.lib.utils.TimestampPeriod attribute)@\spxentry{COMPARE}\spxextra{weatherDB.lib.utils.TimestampPeriod attribute}}

\begin{fulllineitems}
\phantomsection\label{\detokenize{weatherDB.lib:weatherDB.lib.utils.TimestampPeriod.COMPARE}}\pysigline{\sphinxbfcode{\sphinxupquote{COMPARE}}\sphinxbfcode{\sphinxupquote{\DUrole{w}{  }\DUrole{p}{=}\DUrole{w}{  }\{\textquotesingle{}inner\textquotesingle{}: \{0: \textless{}built\sphinxhyphen{}in function max\textgreater{}, 1: \textless{}built\sphinxhyphen{}in function min\textgreater{}\}, \textquotesingle{}outer\textquotesingle{}: \{0: \textless{}built\sphinxhyphen{}in function min\textgreater{}, 1: \textless{}built\sphinxhyphen{}in function max\textgreater{}\}\}}}}
\end{fulllineitems}

\index{\_\_init\_\_() (weatherDB.lib.utils.TimestampPeriod method)@\spxentry{\_\_init\_\_()}\spxextra{weatherDB.lib.utils.TimestampPeriod method}}

\begin{fulllineitems}
\phantomsection\label{\detokenize{weatherDB.lib:weatherDB.lib.utils.TimestampPeriod.__init__}}\pysiglinewithargsret{\sphinxbfcode{\sphinxupquote{\_\_init\_\_}}}{\emph{\DUrole{n}{start}}, \emph{\DUrole{n}{end}}}{}
\sphinxAtStartPar
Initiate a TimestampPeriod.
\begin{quote}\begin{description}
\item[{Parameters}] \leavevmode\begin{itemize}
\item {} 
\sphinxAtStartPar
\sphinxstyleliteralstrong{\sphinxupquote{start}} (\sphinxstyleliteralemphasis{\sphinxupquote{pd.Timestamp}}\sphinxstyleliteralemphasis{\sphinxupquote{ or }}\sphinxstyleliteralemphasis{\sphinxupquote{similar}}) \textendash{} The start of the Period.

\item {} 
\sphinxAtStartPar
\sphinxstyleliteralstrong{\sphinxupquote{end}} (\sphinxstyleliteralemphasis{\sphinxupquote{pd.Timestamp}}\sphinxstyleliteralemphasis{\sphinxupquote{ or }}\sphinxstyleliteralemphasis{\sphinxupquote{similar}}) \textendash{} The end of the Period.

\end{itemize}

\end{description}\end{quote}

\end{fulllineitems}

\index{contains() (weatherDB.lib.utils.TimestampPeriod method)@\spxentry{contains()}\spxextra{weatherDB.lib.utils.TimestampPeriod method}}

\begin{fulllineitems}
\phantomsection\label{\detokenize{weatherDB.lib:weatherDB.lib.utils.TimestampPeriod.contains}}\pysiglinewithargsret{\sphinxbfcode{\sphinxupquote{contains}}}{\emph{\DUrole{n}{other}}}{}
\end{fulllineitems}

\index{copy() (weatherDB.lib.utils.TimestampPeriod method)@\spxentry{copy()}\spxextra{weatherDB.lib.utils.TimestampPeriod method}}

\begin{fulllineitems}
\phantomsection\label{\detokenize{weatherDB.lib:weatherDB.lib.utils.TimestampPeriod.copy}}\pysiglinewithargsret{\sphinxbfcode{\sphinxupquote{copy}}}{}{}
\end{fulllineitems}

\index{get\_period() (weatherDB.lib.utils.TimestampPeriod method)@\spxentry{get\_period()}\spxextra{weatherDB.lib.utils.TimestampPeriod method}}

\begin{fulllineitems}
\phantomsection\label{\detokenize{weatherDB.lib:weatherDB.lib.utils.TimestampPeriod.get_period}}\pysiglinewithargsret{\sphinxbfcode{\sphinxupquote{get\_period}}}{}{}
\end{fulllineitems}

\index{get\_sql\_format\_dict() (weatherDB.lib.utils.TimestampPeriod method)@\spxentry{get\_sql\_format\_dict()}\spxextra{weatherDB.lib.utils.TimestampPeriod method}}

\begin{fulllineitems}
\phantomsection\label{\detokenize{weatherDB.lib:weatherDB.lib.utils.TimestampPeriod.get_sql_format_dict}}\pysiglinewithargsret{\sphinxbfcode{\sphinxupquote{get\_sql\_format\_dict}}}{\emph{\DUrole{n}{format}\DUrole{o}{=}\DUrole{default_value}{"\textquotesingle{}\%Y\%m\%d \%H:\%M\textquotesingle{}"}}}{}
\end{fulllineitems}

\index{has\_NaT() (weatherDB.lib.utils.TimestampPeriod method)@\spxentry{has\_NaT()}\spxextra{weatherDB.lib.utils.TimestampPeriod method}}

\begin{fulllineitems}
\phantomsection\label{\detokenize{weatherDB.lib:weatherDB.lib.utils.TimestampPeriod.has_NaT}}\pysiglinewithargsret{\sphinxbfcode{\sphinxupquote{has\_NaT}}}{}{}
\end{fulllineitems}

\index{has\_only\_NaT() (weatherDB.lib.utils.TimestampPeriod method)@\spxentry{has\_only\_NaT()}\spxextra{weatherDB.lib.utils.TimestampPeriod method}}

\begin{fulllineitems}
\phantomsection\label{\detokenize{weatherDB.lib:weatherDB.lib.utils.TimestampPeriod.has_only_NaT}}\pysiglinewithargsret{\sphinxbfcode{\sphinxupquote{has\_only\_NaT}}}{}{}
\end{fulllineitems}

\index{inside() (weatherDB.lib.utils.TimestampPeriod method)@\spxentry{inside()}\spxextra{weatherDB.lib.utils.TimestampPeriod method}}

\begin{fulllineitems}
\phantomsection\label{\detokenize{weatherDB.lib:weatherDB.lib.utils.TimestampPeriod.inside}}\pysiglinewithargsret{\sphinxbfcode{\sphinxupquote{inside}}}{\emph{\DUrole{n}{other}}}{}
\end{fulllineitems}

\index{is\_empty() (weatherDB.lib.utils.TimestampPeriod method)@\spxentry{is\_empty()}\spxextra{weatherDB.lib.utils.TimestampPeriod method}}

\begin{fulllineitems}
\phantomsection\label{\detokenize{weatherDB.lib:weatherDB.lib.utils.TimestampPeriod.is_empty}}\pysiglinewithargsret{\sphinxbfcode{\sphinxupquote{is\_empty}}}{}{}
\end{fulllineitems}

\index{strftime() (weatherDB.lib.utils.TimestampPeriod method)@\spxentry{strftime()}\spxextra{weatherDB.lib.utils.TimestampPeriod method}}

\begin{fulllineitems}
\phantomsection\label{\detokenize{weatherDB.lib:weatherDB.lib.utils.TimestampPeriod.strftime}}\pysiglinewithargsret{\sphinxbfcode{\sphinxupquote{strftime}}}{\emph{\DUrole{n}{format}\DUrole{o}{=}\DUrole{default_value}{\textquotesingle{}\%Y\sphinxhyphen{}\%m\sphinxhyphen{}\%d \%H:\%M:\%S\textquotesingle{}}}}{}
\end{fulllineitems}

\index{union() (weatherDB.lib.utils.TimestampPeriod method)@\spxentry{union()}\spxextra{weatherDB.lib.utils.TimestampPeriod method}}

\begin{fulllineitems}
\phantomsection\label{\detokenize{weatherDB.lib:weatherDB.lib.utils.TimestampPeriod.union}}\pysiglinewithargsret{\sphinxbfcode{\sphinxupquote{union}}}{\emph{\DUrole{n}{other}}, \emph{\DUrole{n}{how}\DUrole{o}{=}\DUrole{default_value}{\textquotesingle{}inner\textquotesingle{}}}}{}
\sphinxAtStartPar
Unite 2 TimestampPeriods to one.

\sphinxAtStartPar
Compares the Periods and computes a new one.
\begin{quote}\begin{description}
\item[{Parameters}] \leavevmode\begin{itemize}
\item {} 
\sphinxAtStartPar
\sphinxstyleliteralstrong{\sphinxupquote{other}} ({\hyperref[\detokenize{weatherDB.lib:weatherDB.lib.utils.TimestampPeriod}]{\sphinxcrossref{\sphinxstyleliteralemphasis{\sphinxupquote{TimestampPeriod}}}}}) \textendash{} The other TimestampPeriod with whome to compare.

\item {} 
\sphinxAtStartPar
\sphinxstyleliteralstrong{\sphinxupquote{how}} (\sphinxstyleliteralemphasis{\sphinxupquote{str}}\sphinxstyleliteralemphasis{\sphinxupquote{, }}\sphinxstyleliteralemphasis{\sphinxupquote{optional}}) \textendash{} How to compare the 2 TimestampPeriods.
Can be “inner” or “outer”.
“inner”: the maximal Timespan for both is computed.
“outer”: The minimal Timespan for both is computed.
The default is “inner”.

\end{itemize}

\item[{Returns}] \leavevmode
\sphinxAtStartPar
A new TimespanPeriod object uniting both TimestampPeriods.

\item[{Return type}] \leavevmode
\sphinxAtStartPar
{\hyperref[\detokenize{weatherDB.lib:weatherDB.lib.utils.TimestampPeriod}]{\sphinxcrossref{TimestampPeriod}}}

\end{description}\end{quote}

\end{fulllineitems}


\end{fulllineitems}

\index{get\_ftp\_file\_list() (in module weatherDB.lib.utils)@\spxentry{get\_ftp\_file\_list()}\spxextra{in module weatherDB.lib.utils}}

\begin{fulllineitems}
\phantomsection\label{\detokenize{weatherDB.lib:weatherDB.lib.utils.get_ftp_file_list}}\pysiglinewithargsret{\sphinxcode{\sphinxupquote{weatherDB.lib.utils.}}\sphinxbfcode{\sphinxupquote{get\_ftp\_file\_list}}}{\emph{\DUrole{n}{ftp\_conn}}, \emph{\DUrole{n}{ftp\_folders}}}{}
\sphinxAtStartPar
Get a list of files in the folders with their modification dates.
\begin{quote}\begin{description}
\item[{Parameters}] \leavevmode\begin{itemize}
\item {} 
\sphinxAtStartPar
\sphinxstyleliteralstrong{\sphinxupquote{ftp\_conn}} (\sphinxstyleliteralemphasis{\sphinxupquote{ftplib.FTP}}) \textendash{} Ftp connection.

\item {} 
\sphinxAtStartPar
\sphinxstyleliteralstrong{\sphinxupquote{ftp\_folders}} (\sphinxstyleliteralemphasis{\sphinxupquote{list of str}}\sphinxstyleliteralemphasis{\sphinxupquote{ or }}\sphinxstyleliteralemphasis{\sphinxupquote{pathlike object}}) \textendash{} The directories on the ftp server to look for files.

\end{itemize}

\item[{Returns}] \leavevmode
\sphinxAtStartPar
A list of Tuples. Every tuple stands for one file.
The tuple consists of (filepath, modification date).

\item[{Return type}] \leavevmode
\sphinxAtStartPar
list of tuples of strs

\end{description}\end{quote}

\end{fulllineitems}



\paragraph{Subpackages}
\label{\detokenize{weatherDB.lib:subpackages}}

\subparagraph{weatherDB.lib.max\_fun package}
\label{\detokenize{weatherDB.lib.max_fun:weatherdb-lib-max-fun-package}}\label{\detokenize{weatherDB.lib.max_fun::doc}}

\subparagraph{weatherDB.lib.max\_fun.import\_DWD module}
\label{\detokenize{weatherDB.lib.max_fun:module-weatherDB.lib.max_fun.import_DWD}}\label{\detokenize{weatherDB.lib.max_fun:weatherdb-lib-max-fun-import-dwd-module}}\index{module@\spxentry{module}!weatherDB.lib.max\_fun.import\_DWD@\spxentry{weatherDB.lib.max\_fun.import\_DWD}}\index{weatherDB.lib.max\_fun.import\_DWD@\spxentry{weatherDB.lib.max\_fun.import\_DWD}!module@\spxentry{module}}
\sphinxAtStartPar
A collection of functions to import data from the DWD\sphinxhyphen{}CDC Server.
\index{FTP (class in weatherDB.lib.max\_fun.import\_DWD)@\spxentry{FTP}\spxextra{class in weatherDB.lib.max\_fun.import\_DWD}}

\begin{fulllineitems}
\phantomsection\label{\detokenize{weatherDB.lib.max_fun:weatherDB.lib.max_fun.import_DWD.FTP}}\pysiglinewithargsret{\sphinxbfcode{\sphinxupquote{class\DUrole{w}{  }}}\sphinxcode{\sphinxupquote{weatherDB.lib.max\_fun.import\_DWD.}}\sphinxbfcode{\sphinxupquote{FTP}}}{\emph{\DUrole{n}{host=\textquotesingle{}\textquotesingle{}}}, \emph{\DUrole{n}{user=\textquotesingle{}\textquotesingle{}}}, \emph{\DUrole{n}{passwd=\textquotesingle{}\textquotesingle{}}}, \emph{\DUrole{n}{acct=\textquotesingle{}\textquotesingle{}}}, \emph{\DUrole{n}{timeout=\textless{}object object\textgreater{}}}, \emph{\DUrole{n}{source\_address=None}}, \emph{\DUrole{n}{*}}, \emph{\DUrole{n}{encoding=\textquotesingle{}utf\sphinxhyphen{}8\textquotesingle{}}}}{}
\sphinxAtStartPar
Bases: \sphinxcode{\sphinxupquote{ftplib.FTP}}
\index{login() (weatherDB.lib.max\_fun.import\_DWD.FTP method)@\spxentry{login()}\spxextra{weatherDB.lib.max\_fun.import\_DWD.FTP method}}

\begin{fulllineitems}
\phantomsection\label{\detokenize{weatherDB.lib.max_fun:weatherDB.lib.max_fun.import_DWD.FTP.login}}\pysiglinewithargsret{\sphinxbfcode{\sphinxupquote{login}}}{\emph{\DUrole{o}{**}\DUrole{n}{kwargs}}}{}
\sphinxAtStartPar
Login, default anonymous.

\end{fulllineitems}


\end{fulllineitems}

\index{dwd\_id\_to\_str() (in module weatherDB.lib.max\_fun.import\_DWD)@\spxentry{dwd\_id\_to\_str()}\spxextra{in module weatherDB.lib.max\_fun.import\_DWD}}

\begin{fulllineitems}
\phantomsection\label{\detokenize{weatherDB.lib.max_fun:weatherDB.lib.max_fun.import_DWD.dwd_id_to_str}}\pysiglinewithargsret{\sphinxcode{\sphinxupquote{weatherDB.lib.max\_fun.import\_DWD.}}\sphinxbfcode{\sphinxupquote{dwd\_id\_to\_str}}}{\emph{\DUrole{n}{id}}}{}
\sphinxAtStartPar
Convert a station id to normal DWD format as str.
\begin{quote}\begin{description}
\item[{Parameters}] \leavevmode
\sphinxAtStartPar
\sphinxstyleliteralstrong{\sphinxupquote{id}} (\sphinxstyleliteralemphasis{\sphinxupquote{int}}\sphinxstyleliteralemphasis{\sphinxupquote{ or }}\sphinxstyleliteralemphasis{\sphinxupquote{str}}) \textendash{} The id of the station.

\item[{Returns}] \leavevmode
\sphinxAtStartPar
string of normal DWD Station id.

\item[{Return type}] \leavevmode
\sphinxAtStartPar
str

\end{description}\end{quote}

\end{fulllineitems}

\index{get\_dwd\_data() (in module weatherDB.lib.max\_fun.import\_DWD)@\spxentry{get\_dwd\_data()}\spxextra{in module weatherDB.lib.max\_fun.import\_DWD}}

\begin{fulllineitems}
\phantomsection\label{\detokenize{weatherDB.lib.max_fun:weatherDB.lib.max_fun.import_DWD.get_dwd_data}}\pysiglinewithargsret{\sphinxcode{\sphinxupquote{weatherDB.lib.max\_fun.import\_DWD.}}\sphinxbfcode{\sphinxupquote{get\_dwd\_data}}}{\emph{\DUrole{n}{station\_id}}, \emph{\DUrole{n}{ftp\_folder}}}{}
\sphinxAtStartPar
Get the weather data for one station from the DWD server.
\begin{quote}\begin{description}
\item[{Parameters}] \leavevmode\begin{itemize}
\item {} 
\sphinxAtStartPar
\sphinxstyleliteralstrong{\sphinxupquote{station\_id}} (\sphinxstyleliteralemphasis{\sphinxupquote{str}}\sphinxstyleliteralemphasis{\sphinxupquote{ or }}\sphinxstyleliteralemphasis{\sphinxupquote{int}}) \textendash{} Number of the station to get the weather data from.

\item {} 
\sphinxAtStartPar
\sphinxstyleliteralstrong{\sphinxupquote{ftp\_folder}} (\sphinxstyleliteralemphasis{\sphinxupquote{str}}) \textendash{} the base folder where to look for the stations\_id file.
e.g. ftp\_folder = “climate\_environment/CDC/observations\_germany/climate/hourly/precipitation/historical/”.
If the parent folder, where “recent”/”historical” folder is inside, both the historical and recent data gets merged.

\end{itemize}

\item[{Returns}] \leavevmode
\sphinxAtStartPar
The DataFrame of the selected file in the zip folder.

\item[{Return type}] \leavevmode
\sphinxAtStartPar
pandas.DataFrame

\end{description}\end{quote}

\end{fulllineitems}

\index{get\_dwd\_file() (in module weatherDB.lib.max\_fun.import\_DWD)@\spxentry{get\_dwd\_file()}\spxextra{in module weatherDB.lib.max\_fun.import\_DWD}}

\begin{fulllineitems}
\phantomsection\label{\detokenize{weatherDB.lib.max_fun:weatherDB.lib.max_fun.import_DWD.get_dwd_file}}\pysiglinewithargsret{\sphinxcode{\sphinxupquote{weatherDB.lib.max\_fun.import\_DWD.}}\sphinxbfcode{\sphinxupquote{get\_dwd\_file}}}{\emph{\DUrole{n}{zip\_filepath}}}{}
\sphinxAtStartPar
Get a DataFrame from one single (zip\sphinxhyphen{})file from the DWD FTP server.
\begin{quote}\begin{description}
\item[{Parameters}] \leavevmode
\sphinxAtStartPar
\sphinxstyleliteralstrong{\sphinxupquote{zip\_filepath}} (\sphinxstyleliteralemphasis{\sphinxupquote{str}}) \textendash{} 
\sphinxAtStartPar
Path to the file on the server. e.g.
\begin{itemize}
\item {} 
\sphinxAtStartPar
”/climate\_environment/CDC/observations\_germany/climate/10\_minutes/air\_temperature/recent/10minutenwerte\_TU\_00044\_akt.zip”

\item {} 
\sphinxAtStartPar
”/climate\_environment/CDC/derived\_germany/soil/daily/historical/derived\_germany\_soil\_daily\_historical\_73.txt.gz”

\end{itemize}


\item[{Returns}] \leavevmode
\sphinxAtStartPar
The DataFrame of the selected file in the zip folder.

\item[{Return type}] \leavevmode
\sphinxAtStartPar
pandas.DataFrame

\end{description}\end{quote}

\end{fulllineitems}

\index{get\_dwd\_meta() (in module weatherDB.lib.max\_fun.import\_DWD)@\spxentry{get\_dwd\_meta()}\spxextra{in module weatherDB.lib.max\_fun.import\_DWD}}

\begin{fulllineitems}
\phantomsection\label{\detokenize{weatherDB.lib.max_fun:weatherDB.lib.max_fun.import_DWD.get_dwd_meta}}\pysiglinewithargsret{\sphinxcode{\sphinxupquote{weatherDB.lib.max\_fun.import\_DWD.}}\sphinxbfcode{\sphinxupquote{get\_dwd\_meta}}}{\emph{\DUrole{n}{ftp\_folder}}, \emph{\DUrole{n}{min\_years}\DUrole{o}{=}\DUrole{default_value}{0}}, \emph{\DUrole{n}{max\_hole\_d}\DUrole{o}{=}\DUrole{default_value}{9999}}}{}
\sphinxAtStartPar
Get the meta file from the ftp\_folder on the DWD server.

\sphinxAtStartPar
Downloads the meta file of a given folder.
Corrects the meta file of missing files. So if no file for the station is
in the folder the meta entry gets deleted.
Reset “von\_datum” in meta file if there is a biger gap than max\_hole\_d.
Delets entries with less years than min\_years.
\begin{quote}\begin{description}
\item[{Parameters}] \leavevmode\begin{itemize}
\item {} 
\sphinxAtStartPar
\sphinxstyleliteralstrong{\sphinxupquote{ftp\_folder}} (\sphinxstyleliteralemphasis{\sphinxupquote{str}}) \textendash{} The path to the directory where to search for the meta file.
e.g. “climate\_environment/CDC/observations\_germany/climate/hourly/precipitation/recent/”.

\item {} 
\sphinxAtStartPar
\sphinxstyleliteralstrong{\sphinxupquote{min\_years}} (\sphinxstyleliteralemphasis{\sphinxupquote{int}}\sphinxstyleliteralemphasis{\sphinxupquote{, }}\sphinxstyleliteralemphasis{\sphinxupquote{optional}}) \textendash{} filter the list of stations by a minimum amount of years,
that they have data for. 0 if the data should not get filtered.
Only works if the meta file has a timerange defined,
e.g. in “observations”.
The default is 0.

\item {} 
\sphinxAtStartPar
\sphinxstyleliteralstrong{\sphinxupquote{max\_hole\_d}} (\sphinxstyleliteralemphasis{\sphinxupquote{int}}) \textendash{} The maximum amount of days missing in the data allowed.
If there are several files for one station and the time hole is biger
than this value, the older “von\_datum” is overwriten
in the meta GeoDataFrame.
The default is 2.

\end{itemize}

\item[{Returns}] \leavevmode
\sphinxAtStartPar
a GeoDataFrame of the meta file

\item[{Return type}] \leavevmode
\sphinxAtStartPar
geopandas.GeoDataFrame

\end{description}\end{quote}

\end{fulllineitems}



\chapter{Indices and tables}
\label{\detokenize{index:indices-and-tables}}\begin{itemize}
\item {} 
\sphinxAtStartPar
\DUrole{xref,std,std-ref}{genindex}

\item {} 
\sphinxAtStartPar
\DUrole{xref,std,std-ref}{modindex}

\item {} 
\sphinxAtStartPar
\DUrole{xref,std,std-ref}{search}

\end{itemize}


\renewcommand{\indexname}{Python Module Index}
\begin{sphinxtheindex}
\let\bigletter\sphinxstyleindexlettergroup
\bigletter{w}
\item\relax\sphinxstyleindexentry{weatherDB.broker}\sphinxstyleindexpageref{weatherDB:\detokenize{module-weatherDB.broker}}
\item\relax\sphinxstyleindexentry{weatherDB.lib.connections}\sphinxstyleindexpageref{weatherDB.lib:\detokenize{module-weatherDB.lib.connections}}
\item\relax\sphinxstyleindexentry{weatherDB.lib.max\_fun.import\_DWD}\sphinxstyleindexpageref{weatherDB.lib.max_fun:\detokenize{module-weatherDB.lib.max_fun.import_DWD}}
\item\relax\sphinxstyleindexentry{weatherDB.lib.utils}\sphinxstyleindexpageref{weatherDB.lib:\detokenize{module-weatherDB.lib.utils}}
\item\relax\sphinxstyleindexentry{weatherDB.station}\sphinxstyleindexpageref{weatherDB:\detokenize{module-weatherDB.station}}
\item\relax\sphinxstyleindexentry{weatherDB.stations}\sphinxstyleindexpageref{weatherDB:\detokenize{module-weatherDB.stations}}
\end{sphinxtheindex}

\renewcommand{\indexname}{Index}
\printindex
\end{document}